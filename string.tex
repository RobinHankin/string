\documentclass[prb,preprint]{revtex4-1} 
% The line above defines the type of LaTeX document.
% Note that AJP uses the same style as Phys. Rev. B (prb).

% The % character begins a comment, which continues to the end of the line.

\usepackage{amsmath}  % needed for \tfrac, \bmatrix, etc.
\usepackage{amsfonts} % needed for bold Greek, Fraktur, and blackboard bold
\usepackage{graphicx} % needed for figures


\begin{document}

% Be sure to use the \title, \author, \affiliation, and \abstract macros
% to format your title page.  Don't use lower-level macros to  manually
% adjust the fonts and centering.

\title{Light inextensible strings under tension in the Schwarzschild geometry}
% In a long title you can use \\ to force a line break at a certain location.

\author{Robin K. S. Hankin}
\email{hankin.robin@gmail.com} % optional
\altaffiliation[permanent address: ]{2-14 Wakefield Street, Auckland, New Zealand} % optional second address
% If there were a second author at the same address, we would put another 
% \author{} statement here.  Don't combine multiple authors in a single
% \author statement.
\affiliation{AUT University, 2-14 Wakefield Street, Auckland, New Zealand}
% Please provide a full mailing address here.

% See the REVTeX documentation for more examples of author and affiliation lists.

\date{\today}

\begin{abstract}
Here I analyze the behaviour of light inextensible strings under
tension in the vicinity of a black hole.  Proper length (given by the
spatial component of the Schwarzshild metric) is minimized by the
Euler-Lagrange equations.  Light inextensible taut strings may assume
a range of distinct configurations with an arbitrarily high winding
number around the central singularity, and here I give a range of
numerical solutions and theoretical observations.
\end{abstract}

% AJP requires an abstract for all regular article submissions.
% Abstracts are optional for submissions to the "Notes and Discussions" section.

\maketitle % title page is now complete

\section{Introduction}

In this short paper I consider the behaviour of a light inextensible
string in the exterior Schwarzschild metric.  A light string is not
subject to gravity, and being under tension means that its
configuration is an extremum of proper length.  Of course, no real
string is perfectly weightless or inextensible.  With tension~$T$,
gravitational acceleration $g$ and (proper) mass per (proper) unit
length~$\mu$, then ideal behaviour will be appropriate for
lengthscales $L\lesssim\frac{T}{\mu g}$.  Using Newtonian expressions
for the strength of gravity at the event horizon gives
$L\ll\frac{TGM}{\mu c^2}$.  For the string to behave itself near the
event horizon on lengthscales comparable with the Schwarzschild radius
would require $T\gtrsim\mu c^2$.  Exceeding this criterion would
violate the dominant energy condition (and in this context we note
that transverse waves would propagate at speed~$\sqrt{T/\mu}\sim c$).

Nevertheless, the idealization of a light inextensible string is an
interesting and informative limit: its analysis interprets the concept
``straight line'' and suggests a possible way of manipulating a black
hole.  

The Schwarzschild metric

\begin{equation}\label{schwarzschild}
ds^2= -dt^2\left(1-2M/r\right) +\frac{dr^2}{1-2M/r} + r^2\left(d\theta^2 + \sin^2\theta d\phi^2\right)
\end{equation}

where units in which~$2GM=c=1$ are used, has been known for over a
century and makes a number of counter-intuitive predictions.  Here I
consider the spatial component

\begin{equation}\label{spatial_schwarzschild}
  ds^2= \frac{dr^2}{1-2M/r} + r^2d\phi^2
\end{equation}

confined to equatorial points~($\phi=\pi/2$); we
understand~$1<r<\infty$ throughout.  A string of constant~$r$ will,
trivially, have proper length~$2\pi r$.

\subsection*{Radial strings}

Considering first the radial component we may calculate the
proper length $L\left(r_1,r_2\right)$ of a string stretching
from~$r=r_1\geq 1$ to~$r_2\geq r_1$ at constant $\theta,\phi$:

\begin{equation}\label{radial_string_length}
  \int_{r=r_1}^{r_2}\frac{dr}{\sqrt{1-2M/r}}=
  \left.
  \sqrt{r(r-1)} +\frac{1}{2}\log\left(
  \frac{r+\sqrt{r-1}}{r-\sqrt{r-1}}\right)
  \right|_{r_1}^{r_2}
  \end{equation}

Observe that integration may be performed with a lower limit
of~$r_1=1$ with no difficulty, even though the radial component of the
Schwarzschild metric approaches infinity there.  The proper string
length from~$r=1$ to~$r=1+r_0$, say, is given by elementary calculus
and we can say:

\begin{equation}\label{proper_string_length_large_r}
  L\left(1,1+r_0\right)=
  r_0 + \frac{\log(r_0)}{2}
  + \log(2) -\frac{1}{2} 
  +o(1)\qquad\mbox{as $r_0\longrightarrow\infty$}
\end{equation}


%In[23]:=  p = Assuming[r0>1,Integrate[1/Sqrt[1-1/r],{r,1,1+r0}]] 
%
%                                            r0
%Out[23]= Sqrt[r0 (1 + r0)] + ArcTanh[Sqrt[------]]
%                                          1 + r0
%
%In[24]:= Limit[p -(r0 + Log[2] + 1/2 +Log[r0]/2) , {r0 -> Infinity}] 
%
%Out[24]= {0}


For small values of~$r_0$ we have 

\begin{equation}\label{proper_string_length_small_r}
L\left(1,1+r_0\right) =   \sqrt{r_0}\left(2 - \frac{1}{6}r_0+ o\left(r_0^2\right)\right)\qquad\mbox{as $r_0\longrightarrow 0^+$}
\end{equation}


%var('r,r0')
%def g(r):  return sqrt(r*(r-1)) + log(    (r+sqrt(r-1))/(r-sqrt(r-1))  )/2
% (   (g(1+r0)-g(1))/sqrt(r0)    ).taylor(r0,0,2)

% 3/40*r0^2 - 1/6*r0 + 2




\subsection{Radial and angular components}

Taut strings with a metric corresponding
to~\ref{spatial_schwarzschild} may be visualized using the device of
{\em Flamm's paraboloid\,}~\cite{flamm1916}
(figure~\ref{flamm_nostring}), in which a 2D surface corresponding
to~$\theta=0$ is embedded into~$\mathbb{R}^3$ in such a way that
distances on the surface correspond to the spatial components of the
Schwarzschild metric.  Note the absence of points within the event
horizon.

Integrating equation~\ref{spatial_schwarzschild} along a path between
two points~$p_1=\left(r_1,\phi_1\right)$
and~$p_2=\left(r_2,\phi_2\right)$ gives the {\em proper} length of the
path, which is the length of a string joining~$p_1$ and~$p_2$.  A taut
but light string from~$p_1$ to~$p_2$ will adopt an extremal-length
path, as measured by equation~\ref{spatial_schwarzschild}.  Such paths
may be found by the calculus of variations; parametrizing a curve in
terms of~$r=r\left(\phi\right)$ gives us a path length of

\begin{equation}
%  \int_{p_1}^{p_2}\sqrt{\frac{\left(r'\right)^2}{1-2M/r} + r^2}d\phi=
  \int_{p_1}^{p_2}\left(\frac{\left(r'\right)^2}{1-2M/r} + r^2\right)^\frac{1}{2}d\phi=
  \int_{p_1}^{p_2}F\left(r,r'\right)d\phi.
\end{equation}

where dashes denote differentiation with respect to~$\phi$.  The
Euler-Lagrange equations for this system are

\begin{equation}
  \frac{d}{d\phi}\frac{\partial F}{\partial r'}-\frac{\partial F}{\partial r}=0.
\end{equation}

This gives, after simplification, an expression for the second
derivative of~$r$:

\begin{equation}\label{rdashdash}
  r''\left(\phi\right) =
  (r-1) + \frac{(4r-3)\left(r'\right)^2}{2r\left(r-1\right)},\qquad r>1
\end{equation}

This differential equation does not appear to have a simple analytical
solution (I present some numerical results below) but nevertheless we
may make several observations.  Equation~\ref{rdashdash} is a
second-order nonlinear ordinary differential equation and thus has two
constants of integration.  Consider a solution where there exists a
point~$\left(r,\phi\right), r>1$ at which~$r'(\phi)=0$.  By Rolle's
theorem, this point must be unique, as~$r''(\phi)>0$, and be the
closest approach of the string to the event horizon.  Further, because
the system is symmetrical under~$\phi\longrightarrow-\phi$, the string
must be symmetrical about this point; and we can see that~$r$ can take
arbitrarily large values as~$\phi\longrightarrow\pm\infty$: the string
has two ``free'' ends which continue to asymptotically flat space.

Solutions for which~$r'(\phi)>0$ whenever~$r>1$ correspond to strings
that have only one free end; the other is tangential to the event
horizon, except in the exceptional case of constant~$\phi$ when the
string is radial for~$r>1$.  The (signed) radius of curvature~$R$ is
readily evaluated:

\begin{equation}\label{roc}
  R = \frac{
    \left(r^2 + \left(r'\right)^2\right)^\frac{3}{2}
  }{
%    \left|
    r^2 + 2\left(r'\right)^2-rr''
%    \right|
  }
  =
  \frac{
    \left(r^2 + \left(r'\right)^2\right)^\frac{3}{2}
  }{
    r-\frac{\left(r'\right)^2}{2\left(r-1\right)}
  }\qquad\mbox{along a taut string}
\end{equation}

which would give~$R$ as a function of~$\left(r,r'\right)$.
Considering~$r'(\phi)=0$ shows that if a string is tangential at a
point with~$r_0>1$, then it has a radius of curvature of~$r_0^2$ at
that point.  Any circular path, including a circular orbit at~$r=r_0$,
has a radius of curvature of $r_0<r_0^2$, showing that circular orbits
cannot follow a taut string.  Also observe that null geodesics have a
radius of curvature of $2r_0^2/3 < r_0^2$, showing that light cannot
follow a taut string either.


% \subsection{Particle paths}
% 
% Free particle paths are given by
% 
% \begin{equation}
% \frac{d^2u}{d\phi^2} + u = \frac{2}{\ell^2} + \frac{3}{2}
% \end{equation}
% 
% (Rindler 11.45) where~$u=r^{-1}<1$ is the inverse radius and $\ell$ is a constant.
% This, in conjunction with equation~\ref{roc}, shows that tangential
% particle paths have a radius of curvature
% of~$\frac{2\ell^2}{1+3\ell^2/r^2}$.  This cannot equal the radius of
% curvature of a tangential string, which is~$r^2$ for any value
% of~$\ell$; thus no tangentially moving particle can follow a string
% path, no matter how fast it moves.

Equation~\ref{roc} also shows that~$R$ becomes infinite, and the
string has a point of inflection,
if~$\left(r'\right)^2=2r\left(r-1\right)$.

\begin{figure}[h!]
\centering
\includegraphics{flamm_nostring.pdf}
\caption{Flamm's paraboloid}
\label{flamm_nostring}
\end{figure}

\begin{figure}[h!]
\centering
\includegraphics{flamm_string.pdf}
\caption{Flamm's paraboloid with a string}
\label{flamm_withstring}
\end{figure}

\section{Results}
Light inextensible strings fall into two distinct classes: those with
two free ends, and those with one free end (the other ``end'' is
tangential to the event horizon).  Results for strings with two free
ends are presented first.


\subsection{Strings with two free ends}

The canonical result is shown in Figure~\ref{flamm_withstring}:
Flamm's paraboloid with a taut string.  Note that an actual 3D model
would require the string to be constrained above and below.

Figure~\ref{closest_approach} shows strings indexed by distance of
approach to the singularity, which occurs at~$\phi=0$.  Strings with a
sufficiently close passage to the event horizon self-intersect;
numerical methods show that the critical value is~$\simeq
1.076109317$, with an uncertainty of 1 in the last digit: if a string
with two free ends passes closer than this, it forms a closed loop.

\begin{figure}[h!]
\centering
\includegraphics{closest_approach.pdf}
\caption{Strings arranged by closest approach to the event horizon}
\label{closest_approach}
\end{figure}

It is possible to parametrize the string in a number of different
ways.  Figure~\ref{angle_at_r_equals_2} shows the same physical
system, but the strings are arranged so that they pass through the
point~$\left(2,0\right)$; the strings are indexed by the
angle~$\alpha$ made with a radius at that point.

\begin{figure}[h!]
\centering
\includegraphics{angle_at_r_equals_2.pdf}
\caption{Strings \label{angle_at_r_equals_2} passing through
  $\left(2,0\right)$, two free ends}
\end{figure}

\subsection{Strings with one free end}


\begin{figure}[h!]
\centering
\includegraphics{one_free_end_r_equals_2.pdf}
\caption{Strings arranged by angle made at $r=2$, one free end}
\label{y}
\end{figure}

\begin{figure}[h!]
\centering
\includegraphics{one_free_end_fixed_EH_intersection.pdf}
\caption{Strings arranged by QFGASDSF}
\label{fixed_EH_intersection}
\end{figure}

\begin{figure}[h!]
\centering
\includegraphics{curvature_switch.pdf}
\caption{Strings arranged by QFGASDSF}
\label{curvature_switch}
\end{figure}



\section{Discussion}


The analysis above was restricted to the case~$r\geq 1$, where it is
possible to hold objects steady.  Of course, once within the event
horizon, even a light inextensible string cannot hold itself steady
(at least in Schwarzschild coordinates) and one would have to account
for the dynamics of thes system.

Although the concept of light inextensible string is not entirely
consistent---tugging one end of a string cannot produce an
instantaneous response at the other end---it is possible to speculate
about what might happen if~$r<1$.

We might consider two spaceships, or perhaps more appropriately
spiders, on identical freefall radial plunge trajectories separated by
a constant angle~$\theta$.  The spiders might indeed pass a light
inextensible string between themselves and reel it in so as to
maintain a tension (but not so tight as to perturb their radial fall);
what happens to the string?  The spiders will lose causal contact with
one another after a finite proper time but before encountering the
singularity, and it is not clear what happens to their string.

Considering Figure~\ref{closest_approach} in which strings under
tension are near the black hole.  One's classical intuition would
suggest that the black hole feels a force to the right, and should
start accelerating.  However, it is not clear by what mechanism such a
force would be transmitted; the situation is steady, so no
gravitational waves are produced; the string is inextensible, so
stores no energy.  Likewise, Figure~\ref{fixed_EH_intersection} would
suggest that angular momentum is transmitted to the black hole, but no
mechanism is apparent.

This work considered only the Schwarzschild metric and generalizing it
to the Kerr metric~\cite{asdfsadfasfd} might be interesting.  

\appendix*   % Omit the * if there's more than one appendix.

\section{Uninteresting stuff}

Appendices are for material that is needed for completeness but
not sufficiently interesting to include in the main body of the paper.  Most
articles don't need any appendices, but feel free to use them when
appropriate.  This sample article needs an appendix only to illustrate how 
to create an appendix.

\section{Conclusions and further work}


\begin{acknowledgments}

We gratefully acknowledge 
\end{acknowledgments}


\begin{thebibliography}{99}
% The numeral (here 99) in curly braces is nominally the number of entries in
% the bibliography. It's supposed to affect the amount of space around the
% numerical labels, so only the number of digits should matter--and even that
% seems to make no discernible difference.

\bibitem{flamm1916} Flamm, L.: Beitr\"{a}ge zur Einsteinschen
  Gravitationstheorie.  Physikalische Zeitschrift XVII, 448–454 (1916)
  
\bibitem{noBIBTeX} Many \LaTeX\ users manage their bibliographic data with 
a tool called BIB\TeX.  Unfortunately, AJP cannot accept BIB\TeX\ files; all 
bibliographic references must be incorporated into the manuscript file
as shown here, at least when you send an editable file for production.

\bibitem{dyson} Freeman J. Dyson, ``Feynman's proof of the Maxwell equations,''
Am. J. Phys. \textbf{58} (3), 209--211.  
% The issue number (3) in this citation is optional, because AJP's pagination 
% is by volume.



\end{thebibliography}

% If your manuscript is conditionally accepted, the editors will ask you to
% submit your editable LaTeX source file.  Before doing so, you should move
% all tables and figure captions to the end, as shown below.  Tables come 
% first, followed by figure captions (with figure inclusions commented-out).
% Figures should be submitted as separate files, collected with the
% LaTeX file into a single .zip archive.

%\newpage   % Start a new page for tables

%\begin{table}[h!]
%\centering
%\caption{Elementary bosons}
%\begin{ruledtabular}
%\begin{tabular}{l c c c c p{5cm}}
%Name & Symbol & Mass (GeV/$c^2$) & Spin & Discovered & Interacts with \\
%\hline
%Photon & $\gamma$ & \ \ 0 & 1 & 1905 & Electrically charged particles \\
%Gluons & $g$ & \ \ 0 & 1 & 1978 & Strongly interacting particles (quarks and gluons) \\
%Weak charged bosons & $W^\pm$ & \ 82 & 1 & 1983 & Quarks, leptons, $W^\pm$, $Z^0$, $\gamma$ \\
%Weak neutral boson & $Z^0$ & \ 91 & 1 & 1983 & Quarks, leptons, $W^\pm$, $Z^0$ \\
%Higgs boson & $H$ & 126 & 0 & 2012 & Massive particles (according to theory) \\
%\end{tabular}
%\end{ruledtabular}
%\label{bosons}
%\end{table}

%\newpage   % Start a new page for figure captions

%\section*{Figure captions}

%\begin{figure}[h!]
%\centering
%\includegraphics[width=5in]{ThreeSunsets.jpg}   % This line stays commented-out
%\caption{Three overlaid sequences of photos of the setting sun, taken
%near the December solstice (left), September equinox (center), and
%June solstice (right), all from the same location at 41$^\circ$ north
%latitude. The time interval between images in each sequence is approximately
%four minutes.}
%\label{sunsets}
%\end{figure}

\end{document}
