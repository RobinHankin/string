\documentclass[prb,preprint]{revtex4-1} 
% The line above defines the type of LaTeX document.
% Note that AJP uses the same style as Phys. Rev. B (prb).

% The % character begins a comment, which continues to the end of the line.

\usepackage{amsmath}  % needed for \tfrac, \bmatrix, etc.
\usepackage{amsfonts} % needed for bold Greek, Fraktur, and blackboard bold
\usepackage{graphicx} % needed for figures

\begin{document}

% Be sure to use the \title, \author, \affiliation, and \abstract macros
% to format your title page.  Don't use lower-level macros to  manually
% adjust the fonts and centering.

\title{Light inextensible strings under tension in the Schwarzschild geometry}
% In a long title you can use \\ to force a line break at a certain location.

\author{Robin K. S. Hankin}
\email{hankin.robin@gmail.com} % optional
\altaffiliation[permanent address: ]{55 Wellesley Street East,
Auckland, New Zealand}
% If there were a second author at the same address, we would put another 
% \author{} statement here.  Don't combine multiple authors in a single
% \author statement.
\affiliation{School of Engineering, Computer and Mathematical Sciences, Auckland University of Technology}
% Please provide a full mailing address here.


% See the REVTeX documentation for more examples of author and affiliation lists.

\date{\today}

\begin{abstract}
Here I analyze the behaviour of stationary, light inextensible strings
under tension in the vicinity of a nonrotating black hole.  Such
strings minimize their proper length, given by integrating the spatial
component of the Schwarzshild metric along the string.  The path is
calculated by numerical solution to the Euler-Lagrange equations.  If
the string is entirely outside the event horizon, its closest approach
is to the singularity is tangential; if the distance of closest
approach is less than about $1.076109317$ Schwarzschild radii, the
string self-intersects.  If the string attains $r=2GM/c^2$, then it
will in general be tangential to the event horizon.  Light taut
strings interpret the concept ``straight line'' and their behaviour
suggests a possible way of manipulating a black hole.
\end{abstract}
% AJP requires an abstract for all regular article submissions.
% Abstracts are optional for submissions to the "Notes and Discussions" section.

\maketitle % title page is now complete


\section{Introduction} % Section titles are automatically converted to all-caps.
% Section numbering is automatic.

\label{intro}
Light inextensible strings are a stalwart feature of elementary
physics.  A light string is not subject to gravity, and being under
tension means that its configuration is an extremum of proper length.
Of course, no real string is perfectly weightless or inextensible.
With tension~$T$, gravitational acceleration $g$ and (proper) mass per
(proper) unit length~$\mu$, then a string will behave as though light
on lengthscales $L\lesssim T/\left(\mu g\right)$.  Using Newtonian
expressions for the strength of gravity at the event horizon of a
black hole gives $L\lesssim TGM/\left(\mu c^2\right)$.  For the string
to behave as though light on lengthscales comparable with the
Schwarzschild radius would require $T\gtrsim\mu c^2$.  Exceeding this
criterion would violate the dominant energy condition (and in this
context we note that transverse waves would propagate at
speed~$\sqrt{T/\mu}\sim c$).  Nevertheless, the idealization of a
string as light and inextensible is an interesting and informative
limit: its analysis interprets the concept ``straight line'' and
suggests a possible way of manipulating a black hole.  Peculiar
counterintuitive behaviour such as inward-directed centrifugal force
in the vicinity of a black hole is well
known~\citep{abramowicz1992,abramowicz1993}, and the analysis of taut
strings furnishes another everyday window into general relativity.
Another possibility, presented by \citet{damour1978} and discussed by
\citet{thorne1986}, is to pass electric current through such taut
strings, an option not pursued here.

The Schwarzschild metric

\begin{equation}\label{schwarzschild}
ds^2= -dt^2\left(1-1/r\right) +\frac{dr^2}{1-1/r} + r^2\left(d\theta^2 + \sin^2\theta d\phi^2\right),
\end{equation}

\noindent where units in which~$2GM=c=1$ are used, has been known for
over a century~\cite{schwarzschild1916}.  Here I consider the spatial
component~$ds^2= dr^2/\left(1-1/r\right) + r^2d\phi^2$, confined to
equatorial points~($\theta=\pi/2$); we understand~$1\leq r<\infty$
throughout.  A circular string at constant~$r$ will, trivially, have
proper length~$2\pi r$.

Considering first the radial component we may calculate the
proper length $L\left(r_1,r_2\right)$ of a string stretching
from~$r=r_1\geq 1$ to~$r_2\geq r_1$ at constant $\theta,\phi$:

\begin{equation}\label{radial_string_length}
  \int_{r=r_1}^{r_2}\frac{dr}{\sqrt{1-1/r}}=
  \left.
  \sqrt{r(r-1)} +\frac{1}{2}\log\left(
  \frac{r+\sqrt{r-1}}{r-\sqrt{r-1}}\right)
  \right|_{r_1}^{r_2}
  \end{equation}

Observe that integration may be performed with a lower limit
of~$r_1=1$ with no difficulty, even though the radial component of the
Schwarzschild metric approaches infinity there.

Elementary methods show that the proper length $\ell$ of radial strings has
some peculiar features.  Firstly, consider the proper length from
$r=1$ to $r=1+w$ where $w>0, \left|w\right|\ll 1$.  Then

\begin{equation}
  \ell=\int_{r=1}^{r=1+w}ds
  =2\sqrt{w} + \frac{1}{3}w^{3/2} + O(w^{5/2})
\end{equation}

This would indicate that, close to the Schwarzschild radius $r=1$,
large (but finite) amounts of string are required to connect two
points with slightly differing radial coordinates.  Passing to the
case $w\gg 1$ we have an asymptotic expansion that begins

\begin{equation}
  \ell  =\frac{1}{2} + \log(2) + w + \frac{1}{8}w^{-1} + O(w^{-2}),
\end{equation}

\noindent suggesting that there is ``extra radial space'', of length
over $1.19$ Schwarzschild radii, sensible at large distances from the
event horizon.  This might suggest that radial strings have longer
proper length than expected.

\subsection{Nonradial string}
Integrating along a path between two
points~$p_1=\left(r_1,\phi_1\right)$ and~$p_2=\left(r_2,\phi_2\right)$
gives the {\em proper} length of the path, which is the length of an
inextensible string joining~$p_1$ and~$p_2$.  A taut but light string
from~$p_1$ to~$p_2$ will adopt an extremal-length path.  Such paths
may be found by the calculus of variations; parametrizing a curve in
terms of~$r=r\left(\phi\right)$ gives us a path length of

\begin{equation}
%  \int_{p_1}^{p_2}\sqrt{\frac{\left(r'\right)^2}{1-2M/r} + r^2}d\phi=
  \int_{p_1}^{p_2}\left(\frac{\left(r'\right)^2}{1-1/r} + r^2\right)^\frac{1}{2}d\phi=
  \int_{p_1}^{p_2}F\left(r,r'\right)d\phi
\end{equation}

\noindent where dashes denote differentiation with respect to~$\phi$.  The
Euler-Lagrange equations for this system,
$\frac{d}{d\phi}\frac{\partial F}{\partial r'}-\frac{\partial
  F}{\partial r}=0$ give, after simplification, an expression for the
second derivative of~$r$:

\begin{equation}\label{rdashdash}
  r''\left(\phi\right) =
  (r-1) + \frac{(4r-3)\left(r'\right)^2}{2r\left(r-1\right)},\qquad r>1.
\end{equation}

Solutions to this differential equation correspond to a taut string
confined to Flamm's paraboloid~\cite{flamm1916}; see
figure~\ref{flamm}.

There does not appear to be a simple analytical solution but
nevertheless we may make several observations.
Equation~\ref{rdashdash} is a second-order nonlinear ordinary
differential equation and thus has two constants of integration.
There are two qualitatively different types of solutions: those with a
tangential point ($r'=0,r>1$), and those without.  Consider first
solutions with such a point: there exists a pair~$\left(r,\phi\right),
r>1$ at which~$r'(\phi)=0$.  By Rolle's theorem, this point must be
unique, as~$r''(\phi)>0$, and be the closest approach of the string to
the event horizon.

Further, because the system is symmetrical
under~$\phi\longrightarrow-\phi$, the string must be symmetrical about
at this point.  The (signed) radius of curvature~$R$ is readily
evaluated:

\begin{equation}\label{roc}
  R = \frac{
    \left(r^2 + \left(r'\right)^2\right)^\frac{3}{2}
  }{
%    \left|
    r^2 + 2\left(r'\right)^2-rr''
%    \right|
  }
  =
  \frac{
    \left(r^2 + \left(r'\right)^2\right)^\frac{3}{2}
  }{
    r-\frac{\left(r'\right)^2}{2\left(r-1\right)}
  }\qquad\mbox{along a taut string}
\end{equation}

\noindent which would give~$R$ as a function of~$\left(r,r'\right)$.
Considering~$r'(\phi)=0$ shows that if a string is tangential at a
point with~$r_0>1$, then it has a radius of curvature of~$r_0^2$ at
that point.  Any circular path, including a circular orbit at~$r=r_0$,
has a radius of curvature of $r_0<r_0^2$, showing that circular orbits
cannot follow a taut string at closest approach.  Also observe that
null geodesics have a radius of curvature of $2r_0^2/3 < r_0^2$,
showing that light cannot follow a taut string at closest approach
either.  The string has a point of inflection
if~$\left(r'\right)^2=2r(r-1)$.

Compare the above analysis with a taut {\em heavy} string which
adopts, at least locally, a catenary configuration at closest
approach~\cite{nguyen2007}.  Such a configuration is manifestly convex
to the singularity, in contrast to light strings which are concave.

If, conversely, $r'(\phi)>0$ whenever~$r>1$, then the string has only
one free end; the other is tangential to the event horizon, except in
the exceptional case of constant~$\phi$ when the string is radial
for~$r>1$.

\section{Numerical results}

Here I solve equation~\ref{rdashdash} using the \verb+deSolve+ package
for ordinary differential equations~\cite{soetart2010}.  Solutions
fall into two distinct classes: those with $r>1$ at all points, and
those which attain $r=1$, in which case the string is tangential to
the event horizon, unless it is radial.

Taking strings for which $r>1$ first,
figure~\ref{closest_approach_non_self_intersecting} shows a sequence
of non-self-intersecting numerical solutions to
equation~\ref{rdashdash}, rotated so that the tangential point of
closest approach occurs at~$\phi=0$.
Figure~\ref{closest_approach_self_intersecting} shows the
corresponding diagram for self-interesecting strings
(figure~\ref{light_closest_approach} shows comparable null geodesic
diagram for comparison).  A bisection technique reveals that if the
closest approach is greater than a critical value of~$\simeq
1.076109317(1)$, the string does not self-intersect; if the closest
approach is less than this value, then the string crosses itself
at~$\phi=\pi/2$.  As discussed above, if $\left(r'\right)^2=2r(r-1)$,
the string has a point of inflection at which point the radius of
curvature becomes infinite.  Figure~\ref{curvature_switch} shows
non-self intersecting strings rotated so that the point of inflection
occurs at~$\phi=0$; we see that the strings have a finite length in
which convex to the origin and two semi-infinite stretches in which
the curvature goes the other way.

The case where the string is tangent to the event horizon at~$\phi=0$,
including a radial string, is shown in
Figure~\ref{fixed_EH_intersection}.

\section{Discussion}

Light inextensible strings under tension adopt non-trivial
configurations including self-intersecting paths.  The path of such
strings furnishes an interpretation of the concept ``straight line''
in general relativity, others being null geodesics and world lines of
massive objects.  Comparing a world line with that of nearby taut
strings thus furnishes another insight into path curvature,
complementing the analysis of Abramowicz~\cite{abramowicz1992}.

In Figure~\ref{closest_approach_non_self_intersecting}, one's
classical intuition would suggest that the black hole feels a force to
the left, and should start accelerating.  However, it is not clear by
what mechanism such a force would be transmitted: the situation is
steady, so no gravitational waves are produced, and the string is
inextensible so stores no energy.  Further work might include analysis
of taut strings near a spinning black hole.

%%Vancouver style references.
\bibliographystyle{model3-num-names}
\bibliography{stringrefs}

\section*{Supplementary Material}

Calculations are performed using the R programming language~\citep{rcore2018};
source code for the figures is available at
\\
\\
\url{https://github.com/RobinHankin/string.git}


\section*{Figures}

\begin{figure}[h!] % Created by "flamm.R"
\centering
\includegraphics[width=170mm]{flamm_string.pdf}
\caption{Perspective view of Flamm's paraboloid with superimposed minimal-length
  path corresponding to a taut, light string}
\label{flamm}
\end{figure}

\begin{figure}[p] % Created by "closest_approach.R"
\centering
\includegraphics[width=170mm]{closest_approach_nonselfintersecting.pdf}
\caption{Non-self intersecting strings arranged by distance~$d$ of
  closest approach to the event horizon, occurring tangentially
  at~$\phi=0$}
\label{closest_approach_non_self_intersecting}
\end{figure}

\begin{figure}[p] % created by "closest_approach2.R"
\centering
\includegraphics[width=170mm]{closest_approach_selfintersecting.pdf}
\caption{Self-intersecting strings arranged by closest approach to the
  event horizon which occurs at $\phi=0$.  Strings shown for $\phi\geq
  0$}
\label{closest_approach_self_intersecting}
\end{figure}

\begin{figure}[p] % Created by "light_closest_approach.R"
\centering
\includegraphics[width=170mm]{light_closest_approach.pdf}
\caption{Null geodesics in the Schwarzschild geometry passing
  through~$(r,0)$, at which point they are tangential.  Note the
  differences between these curves and the taut strings shown
  elsewhere: unlike taug strings, null geodesics have positive
  curvature everywhere, may cross the event horizon inwards, and are
  never tangential to the event horizon}
\label{light_closest_approach}
\end{figure}

\begin{figure}[p]
\centering
\includegraphics{curvature_switch.pdf} % created by "radius_of_curvature_switch.R"
\caption{Strings arranged by point of curvature switch, which occurs
  at~$\phi=0$.  Colour changes from black to red at points of
  infinite~$R$; black indicates convex to the origin and red indicates
  concave regions}
\label{curvature_switch}
\end{figure}

\begin{figure}[p]
\centering
\includegraphics{angle_at_r_equals_2.pdf} % created by "angle_at_r_equals_2.R"
\caption{Strings passing through $(2,0)$}
\label{strings_r_equals_2}
\end{figure}

\begin{figure}[p]
\centering
\includegraphics{light_start_at_r_equals_2.pdf} % created by "angle_start_at_r_equals_2.R"
\caption{Null geodesics passing through $(2,0)$, shown for $0\leq\phi\leq 2\pi$}
\label{light_r_equals_2}
\end{figure}

\begin{figure}[p] % created by "one_free_end_r_equals_2.R"
\centering
\includegraphics[width=170mm]{one_free_end_r_equals_2.pdf}
\caption{Strings passing through $(2,0)$ tangential to event horizon}
\label{one_free_end_r_equals_2}
\end{figure}

\begin{figure}[p] % created by "one_free_end_fixed_EH_intersection.R"
\centering
\includegraphics[width=170mm]{one_free_end_fixed_EH_intersection.pdf}
\caption{Strings tangential to event horizon}
\label{fixed_EH_intersection}
\end{figure}



\end{document}
