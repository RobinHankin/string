%% This is file `rinp-template.tex',
%% 
%% Copyright 2016 Elsevier Ltd
%% 
%% This file is part of the 'Elsarticle Bundle'.
%% ---------------------------------------------
%% 
%% It may be distributed under the conditions of the LaTeX Project Public
%% License, either version 1.2 of this license or (at your option) any
%% later version.  The latest version of this license is in
%%    http://www.latex-project.org/lppl.txt
%% and version 1.2 or later is part of all distributions of LaTeX
%% version 1999/12/01 or later.
%% 
%% The list of all files belonging to the 'Elsarticle Bundle' is
%% given in the file `manifest.txt'.
%% 
%% Template article for Elsevier's document class `elsarticle'
%% with harvard style bibliographic references
%%
%% $Id: $
%%
%% Use the option review to obtain double line spacing
%\documentclass[times,review,preprint,authoryear]{elsarticle}

%% Use the options `twocolumn,final' to obtain the final layout
%% Use longtitle option to break abstract to multiple pages if overfull.
%% For Review pdf (With double line spacing)
%\documentclass[times,twocolumn,review]{elsarticle}
%% For abstracts longer than one page.
%\documentclass[times,twocolumn,review,longtitle]{elsarticle}
%% For Review pdf without preprint line
%\documentclass[times,twocolumn,review,nopreprintline]{elsarticle}
%% Final pdf
\documentclass[times,twocolumn,final]{elsarticle}
%%
%\documentclass[times,twocolumn,final,longtitle]{elsarticle}
%%


%% Stylefile to load RINP template
\usepackage{rinp}
\usepackage{framed,multirow}

%% The amssymb package provides various useful mathematical symbols
\usepackage{amssymb}
\usepackage{latexsym}

% Following three lines are needed for this document.
% If you are not loading colors or url, then these are
% not required.
\usepackage{url}
\usepackage{xcolor}
\definecolor{newcolor}{rgb}{.8,.349,.1}

\journal{Results in Physics}

\begin{document}

\verso{Robin Hankin}

\begin{frontmatter}

\title{Light inextensible strings under tension in the Schwarzschild geometry}

\author[1]{Robin \snm{Hankin}\corref{cor1}}
\cortext[cor1]{Corresponding author: 
  Tel.: +64-09-921-9999;  
  fax: +64-09-921-9998;}

\address[1]{Auckland University of Technology, 2-14 Wakefield Street,
  Auckland 1142, New Zealand}


\received{1 May 2013}
\finalform{10 May 2013}
\accepted{13 May 2013}
\availableonline{15 May 2013}
\communicated{R. Hankin}


\begin{abstract}
%%%
Here I analyze the behaviour of light inextensible strings under
tension in the vicinity of a black hole.  Proper length (given by the
spatial component of the Schwarzshild metric) is minimized by the
Euler-Lagrange equations.  Light inextensible taut strings may assume
a range of distinct configurations with an arbitrarily high winding
number around the central singularity, and here I give a range of
numerical solutions and theoretical observations.
%%%%
\end{abstract}

\begin{keyword}
%% MSC codes here, in the form: \MSC code \sep code
%% or \MSC[2008] code \sep code (2000 is the default)
\MSC 83C57\sep 35Q75
%% Keywords
\KWD Black Holes\sep Light inextensible string
\end{keyword}

\end{frontmatter}

%\linenumbers

%% main text
\section{Introduction}
\label{intro}
In this short paper I consider the behaviour of a light inextensible
string in the exterior Schwarzschild metric.  A light string is not
subject to gravity, and being under tension means that its
configuration is an extremum of proper length.  Of course, no real
string is perfectly weightless or inextensible.  With tension~$T$,
gravitational acceleration $g$ and (proper) mass per (proper) unit
length~$\mu$, then ideal behaviour will be appropriate for
lengthscales $L\lesssim\frac{T}{\mu g}$.  Using Newtonian expressions
for the strength of gravity at the event horizon gives
$L\ll\frac{TGM}{\mu c^2}$.  For the string to behave itself near the
event horizon on lengthscales comparable with the Schwarzschild radius
would require $T\gtrsim\mu c^2$.  Exceeding this criterion would
violate the dominant energy condition (and in this context we note
that transverse waves would propagate at speed~$\sqrt{T/\mu}\sim c$).

Nevertheless, the idealization of a light inextensible string is an
interesting and informative limit: its analysis interprets the concept
``straight line'' and suggests a possible way of manipulating a black
hole.  The Schwarzschild metric

\begin{equation}\label{schwarzschild}
ds^2= -dt^2\left(1-1/r\right) +\frac{dr^2}{1-1/r} + r^2\left(d\theta^2 + \sin^2\theta d\phi^2\right)
\end{equation}

where units in which~$2GM=c=1$ are used, has been known for over a
century and makes a number of counter-intuitive predictions.  Here I
consider the spatial component~$ds^2= \frac{dr^2}{1-1/r} +
r^2d\phi^2$, confined to equatorial points~($\phi=\pi/2$); we
understand~$1<r<\infty$ throughout.  A string of constant~$r$ will,
trivially, have proper length~$2\pi r$.

Considering first the radial component we may calculate the
proper length $L\left(r_1,r_2\right)$ of a string stretching
from~$r=r_1\geq 1$ to~$r_2\geq r_1$ at constant $\theta,\phi$:

\begin{equation}\label{radial_string_length}
  \int_{r=r_1}^{r_2}\frac{dr}{\sqrt{1-2M/r}}=
  \left.
  \sqrt{r(r-1)} +\frac{1}{2}\log\left(
  \frac{r+\sqrt{r-1}}{r-\sqrt{r-1}}\right)
  \right|_{r_1}^{r_2}
  \end{equation}

Observe that integration may be performed with a lower limit
of~$r_1=1$ with no difficulty, even though the radial component of the
Schwarzschild metric approaches infinity there.  

\subsection{Radial and angular components}

Integrating along a path between two
points~$p_1=\left(r_1,\phi_1\right)$ and~$p_2=\left(r_2,\phi_2\right)$
gives the {\em proper} length of the path, which is the length of a
string joining~$p_1$ and~$p_2$.  A taut but light string from~$p_1$
to~$p_2$ will adopt an extremal-length path.  Such paths may be found
by the calculus of variations; parametrizing a curve in terms
of~$r=r\left(\phi\right)$ gives us a path length of

\begin{equation}
%  \int_{p_1}^{p_2}\sqrt{\frac{\left(r'\right)^2}{1-2M/r} + r^2}d\phi=
  \int_{p_1}^{p_2}\left(\frac{\left(r'\right)^2}{1-2M/r} + r^2\right)^\frac{1}{2}d\phi=
  \int_{p_1}^{p_2}F\left(r,r'\right)d\phi.
\end{equation}

where dashes denote differentiation with respect to~$\phi$.  The
Euler-Lagrange equations for this system,
$\frac{d}{d\phi}\frac{\partial F}{\partial r'}-\frac{\partial
  F}{\partial r}=0$ give, after simplification, an expression for the
second derivative of~$r$:

\begin{equation}\label{rdashdash}
  r''\left(\phi\right) =
  (r-1) + \frac{(4r-3)\left(r'\right)^2}{2r\left(r-1\right)},\qquad r>1
\end{equation}

This differential equation does not appear to have a simple analytical
solution; I present some numerical results below.  Solutions for
which~$r'(\phi)>0$ whenever~$r>1$ correspond to strings that have only
one free end; the other is tangential to the event horizon, except in
the exceptional case of constant~$\phi$ when the string is radial
for~$r>1$.  It is straightforward to show that the radius of curvature
of a taut string becomes infinite, and the string has a point of
inflection, if~$\left(r'\right)^2=2r\left(r-1\right)$.

\section{Results}
Light inextensible strings fall into two distinct classes: those with
two free ends, and those with one free end (the other ``end'' is
tangential to the event horizon).  

Figure~\ref{closest_approach_non_self_intersecting} shows
nonselfintersecting strings indexed by distance of approach to the
singularity, which occurs at~$\phi=0$.  Strings with a sufficiently
close passage to the event horizon self-intersect; numerical methods
show that the critical value is~$\simeq 1.076109317$, with an
uncertainty of 1 in the last digit: if a string with two free ends
passes closer than this, it forms a closed loop.
Figure~\ref{closest_approach_non_self_intersecting} shows some
non-self-intersecting strings and
figure~\ref{closest_approach_self_intersecting} shows some strings
that self-intersect.

\begin{figure}[h!]
\centering
\includegraphics[width=70mm]{closest_approach_nonselfintersecting.pdf}
\caption{Non-self intersecting strings arranged by distance~$d$ of
  closest approach to the event horizon, occurring at~$\phi=0$}
\label{closest_approach_non_self_intersecting}
\end{figure}

\begin{figure}[h!]
\centering
\includegraphics[width=70mm]{closest_approach_selfintersecting.pdf}
\caption{Strings arranged by closest approach to the event horizon:
 self-intersecting strings shown for $\phi\geq 0$}
\label{closest_approach_self_intersecting}
\end{figure}

Figure~\ref{one_free_end_r_equals_2} shows some strings that pass
through~$\left(2,0\right)$ and tangent to the event horizon.

\subsection{Strings with one free end}

\begin{figure}[h!]
\centering
\includegraphics[width=70mm]{one_free_end_r_equals_2.pdf}
\caption{Strings arranged by angle made at $r=2$, one free end}
\label{one_free_end}
\end{figure}


\section{Discussion}


The analysis above was restricted to the case~$r\geq 1$, where it is
possible to hold objects steady.  Of course, once within the event
horizon, even a light inextensible string cannot hold itself steady
(at least in Schwarzschild coordinates) and one would have to account
for the dynamics of thes system.

In figure~\ref{closest_approach_non_self_intersecting}, one's
classical intuition would suggest that the black hole feels a force to
the right, and should start accelerating.  However, it is not clear by
what mechanism such a force would be transmitted; the situation is
steady, so no gravitational waves are produced; the string is
inextensible, so stores no energy.  Likewise,
Figure~\ref{one_free_end} would suggest that angular momentum is
transmitted to the black hole, but no mechanism is apparent.

This work considered only the Schwarzschild metric and generalizing it
to the Kerr metric~\cite{asdfsadfasfd} might be interesting.  


%%Vancouver style references.
\bibliographystyle{model3-num-names}
\bibliography{refs}


\end{document}

%%
