%% This is file `rinp-template.tex',
%% 
%% Copyright 2016 Elsevier Ltd
%% 
%% This file is part of the 'Elsarticle Bundle'.
%% ---------------------------------------------
%% 
%% It may be distributed under the conditions of the LaTeX Project Public
%% License, either version 1.2 of this license or (at your option) any
%% later version.  The latest version of this license is in
%%    http://www.latex-project.org/lppl.txt
%% and version 1.2 or later is part of all distributions of LaTeX
%% version 1999/12/01 or later.
%% 
%% The list of all files belonging to the 'Elsarticle Bundle' is
%% given in the file `manifest.txt'.
%% 
%% Template article for Elsevier's document class `elsarticle'
%% with harvard style bibliographic references
%%
%% $Id: $
%%
%% Use the option review to obtain double line spacing
%\documentclass[times,review,preprint,authoryear]{elsarticle}

%% Use the options `twocolumn,final' to obtain the final layout
%% Use longtitle option to break abstract to multiple pages if overfull.
%% For Review pdf (With double line spacing)
%\documentclass[times,twocolumn,review]{elsarticle}
%% For abstracts longer than one page.
%\documentclass[times,twocolumn,review,longtitle]{elsarticle}
%% For Review pdf without preprint line
%\documentclass[times,twocolumn,review,nopreprintline]{elsarticle}
%% Final pdf
\documentclass[times,twocolumn,final]{elsarticle}
%%
%\documentclass[times,twocolumn,final,longtitle]{elsarticle}
%%


%% Stylefile to load RINP template
\usepackage{rinp}
\usepackage{framed,multirow}

%% The amssymb package provides various useful mathematical symbols
\usepackage{amssymb}
\usepackage{latexsym}

% Following three lines are needed for this document.
% If you are not loading colors or url, then these are
% not required.
\usepackage{url}
\usepackage{xcolor}
\definecolor{newcolor}{rgb}{.8,.349,.1}

\journal{Results in Physics}

\begin{document}

\verso{Robin Hankin}

\begin{frontmatter}

\title{Light inextensible strings under tension in the Schwarzschild geometry}

\author[1]{Robin \snm{Hankin}\corref{cor1}}
\cortext[cor1]{Corresponding author: 
  Tel.: +64-09-921-9999;  
  fax: +64-09-921-9998;}

\address[1]{Auckland University of Technology, 2-14 Wakefield Street,
  Auckland 1142, New Zealand}


\received{day month year}
\finalform{day month year}
\accepted{day month year}
\availableonline{day month year}
\communicated{R. Hankin}


\begin{abstract}
%%%
Here I analyze the behaviour of stationary, light inextensible strings
under tension in the vicinity of a nonrotating black hole.  Such
strings minimize their proper length, given by integrating the spatial
component of the Schwarzshild metric along the string.  The path is
calculated by numerical solution to the Euler-Lagrange equations.  If
the string is entirely outside the event horizon, its closest approach
is to the singularity is tangential; if the distance of closest
approach is less than about $1.076109317$ Schwarzschild radii, the
string self-intersects.  If the string attains $r=2GM$, then it will
in general be tangential to the event horizon at that point.  Light
taut strings interpret the concept ``straight line'' and their
behaviour suggests a possible way of manipulating a black hole.
%%%%
\end{abstract}

\begin{keyword}
%% MSC codes here, in the form: \MSC code \sep code
%% or \MSC[2008] code \sep code (2000 is the default)
\MSC 83C57\sep 35Q75
%% Keywords
\KWD Black holes\sep Light inextensible string
\end{keyword}

\end{frontmatter}

%\linenumbers

%% main text
\section{Introduction}
\label{intro}
Light inextensible strings are a stalwart feature of elementary
physics.  A light string is not subject to gravity, and being under
tension means that its configuration is an extremum of proper length.
Of course, no real string is perfectly weightless or inextensible.
With tension~$T$, gravitational acceleration $g$ and (proper) mass per
(proper) unit length~$\mu$, then a string will behave as though light
on lengthscales $L\lesssim T/\left(\mu g\right)$.  Using Newtonian
expressions for the strength of gravity at the event horizon of a
black hole gives $L\lesssim TGM/\left(\mu c^2\right)$.  For the string
to behave as though light on lengthscales comparable with the
Schwarzschild radius would require $T\gtrsim\mu c^2$.  Exceeding this
criterion would violate the dominant energy condition (and in this
context we note that transverse waves would propagate at
speed~$\sqrt{T/\mu}\sim c$).  Nevertheless, the idealization of a
light inextensible string is an interesting and informative limit: its
analysis interprets the concept ``straight line'' and suggests a
possible way of manipulating a black hole.  The Schwarzschild metric

\begin{equation}\label{schwarzschild}
ds^2= -dt^2\left(1-1/r\right) +\frac{dr^2}{1-1/r} + r^2\left(d\theta^2 + \sin^2\theta d\phi^2\right),
\end{equation}

\noindent where units in which~$2GM=c=1$ are used, has been known for
over a century~\cite{schwarzschild1916} and makes a number of
counter-intuitive predictions.  Here I consider the spatial
component~$ds^2= dr^2/\left(1-1/r\right) + r^2d\phi^2$, confined to
equatorial points~($\theta=\pi/2$); we understand~$1\leq r<\infty$
throughout.  A circular string at constant~$r$ will, trivially, have
proper length~$2\pi r$.

Considering first the radial component we may calculate the
proper length $L\left(r_1,r_2\right)$ of a string stretching
from~$r=r_1\geq 1$ to~$r_2\geq r_1$ at constant $\theta,\phi$:

\begin{equation}\label{radial_string_length}
  \int_{r=r_1}^{r_2}\frac{dr}{\sqrt{1-1/r}}=
  \left.
  \sqrt{r(r-1)} +\frac{1}{2}\log\left(
  \frac{r+\sqrt{r-1}}{r-\sqrt{r-1}}\right)
  \right|_{r_1}^{r_2}
  \end{equation}

Observe that integration may be performed with a lower limit
of~$r_1=1$ with no difficulty, even though the radial component of the
Schwarzschild metric approaches infinity there.  

Integrating along a path between two
points~$p_1=\left(r_1,\phi_1\right)$ and~$p_2=\left(r_2,\phi_2\right)$
gives the {\em proper} length of the path, which is the length of a
string joining~$p_1$ and~$p_2$.  A taut but light string from~$p_1$
to~$p_2$ will adopt an extremal-length path.  Such paths may be found
by the calculus of variations; parametrizing a curve in terms
of~$r=r\left(\phi\right)$ gives us a path length of

\begin{equation}
%  \int_{p_1}^{p_2}\sqrt{\frac{\left(r'\right)^2}{1-2M/r} + r^2}d\phi=
  \int_{p_1}^{p_2}\left(\frac{\left(r'\right)^2}{1-1/r} + r^2\right)^\frac{1}{2}d\phi=
  \int_{p_1}^{p_2}F\left(r,r'\right)d\phi
\end{equation}

\noindent where dashes denote differentiation with respect to~$\phi$.  The
Euler-Lagrange equations for this system,
$\frac{d}{d\phi}\frac{\partial F}{\partial r'}-\frac{\partial
  F}{\partial r}=0$ give, after simplification, an expression for the
second derivative of~$r$:

\begin{equation}\label{rdashdash}
  r''\left(\phi\right) =
  (r-1) + \frac{(4r-3)\left(r'\right)^2}{2r\left(r-1\right)},\qquad r>1.
\end{equation}

This differential equation does not appear to have a simple analytical
solution; I present some numerical results below.  It is
straightforward to show that the radius of curvature of a taut string
becomes infinite, and the string has a point of inflection,
if~$\left(r'\right)^2=2r\left(r-1\right)$.

\section{Results}
Solutions to equation~\ref{rdashdash} fall into two distinct classes:
those with $r>1$ at all points, and those which attain $r=1$, in which
case the string is tangential to the event horizon, in general.
Considering first cases with $r>1$, numerical methods show that if the
closest approach is greater than a critical value of~$\simeq
1.076109317(1)$, the string does not self-intersect
(Figure~\ref{closest_approach_non_self_intersecting}); if the closest
approach is less than this value, the then string crosses itself
at~$\phi=\pi/2$ (Figure~\ref{closest_approach_self_intersecting}).
The case where the string is tangent to the event horizon
at~$r=1,\phi=0$ is shown in Figure~\ref{fixed_EH_intersection}.

\begin{figure}[h!] % Created by "closest_approach.R"
\centering
\includegraphics[width=70mm]{closest_approach_nonselfintersecting.pdf}
\caption{Non-self intersecting strings arranged by distance~$d$ of
  closest approach to the event horizon, occurring tangentially
  at~$\phi=0$}
\label{closest_approach_non_self_intersecting}
\end{figure}

\begin{figure}[h!] % created by "closest_approach2.R"
\centering
\includegraphics[width=70mm]{closest_approach_selfintersecting.pdf}
\caption{Strings arranged by closest approach to the event horizon:
 self-intersecting strings shown for $\phi\geq 0$}
\label{closest_approach_self_intersecting}
\end{figure}

\begin{figure}[h!] % created by "one_free_end_fixed_EH_intersection.R"
\centering
\includegraphics[width=70mm]{one_free_end_fixed_EH_intersection.pdf}
\caption{Strings tangential to event horizon}
\label{fixed_EH_intersection}
\end{figure}

\section{Discussion}

Light inextensible strings under tension adopt non-trivial
configurations including self-intersecting paths.  The path of such
strings furnishes an interpretation of the concept ``straight line''
in general relativity, others being null geodesics and world lines of
massive objects.  Comparing a world line with that of nearby taut
strings thus furnishes another insight into path curvature,
complementing the analysis of Abramowicz~\cite{abramowicz1992}.

In Figure~\ref{closest_approach_non_self_intersecting}, one's
classical intuition would suggest that the black hole feels a force to
the left, and should start accelerating.  However, it is not clear by
what mechanism such a force would be transmitted: the situation is
steady, so no gravitational waves are produced, and the string is
inextensible so stores no energy.

%%Vancouver style references.
\bibliographystyle{model3-num-names}
\bibliography{refs}

\section*{Supplementary Material}

Calculations are performed using the R programming language; source
code for the figures is available at
\\
\\
\url{https://github.com/RobinHankin/string.git}

\end{document}

%%
