\documentclass[review]{elsarticle}

\usepackage{lineno,hyperref}
\usepackage{amsmath}
\usepackage{amssymb}
\usepackage{booktabs}
\usepackage{units}
\usepackage{xcolor}
\modulolinenumbers[5]

\journal{The Physics Teacher}

\bibliographystyle{jss}

\begin{document}
\begin{frontmatter}
\title{Light inextensible strings under tension in the Schwarzschild geometry}

\begin{abstract}

Light inextensible string under tension is a stalwart feature of
elementary physics.  Here I show how considering such string in the
vicinity of a black hole, together with computer algebra systems, can
generate insight into the Schwarzschild geometry that is accessible to
undergraduates.  Light taut strings minimize their proper length,
given by integrating the spatial component of the Schwarzschild metric
along the string.  The path itself is given by straightforward
numerical solution to the Euler-Lagrange equations.  If the string is
entirely outside the event horizon, its closest approach to the
singularity is tangential.  At this point the string is {\em visibly}
curved, surely a memorable and informative insight.  The geometry of
the Schwarzschild metric induces some peculiar nonlocal behaviour: if
the distance of closest approach is less than about $1.0761\cdot 2M$,
the string self-intersects, even though it is everywhere under
tension.  Light taut strings furnish a third interpretation of the
concept ``straight line'', the other two being null geodesics and
free-fall world lines.


% \PACS{PACS code1 \and PACS code2 \and more}
% \subclass{MSC code1 \and MSC code2 \and more}
\end{abstract}

\begin{keyword}
Black holes\sep light inextensible string\sep minimal-length path
\end{keyword}

\end{frontmatter}

\linenumbers
\section{Introduction}

General relativity is a topic of long-standing fascination for
undergraduate physics students~\cite{christensen2012}.  Here I show
how numerical methods, including computer algebra systems, can be used
to interpret the concept of ``straight line'' in the vicinity of a
black hole.  The analysis would be suitable for an undergraduate
homework problem~\cite{romano2019}.

A light string's being under tension means that its path will be an
extremum of proper length: the calculus of variations proves that the
path is straight in the sense of zero curvature.  In terrestrial
gravity, standard $25\times 10^{-6}\unit{kg}\cdot\meter^{-1}$ cotton
sewing thread at its breaking strain of about $40\unit{N}$ would have
a sag of~$\sim 6\times 10^{-6}\meter$ over a $1\meter$ horizontal
scale: barely perceptible.  How does such a quotidian observation
translate into the vicinity of a black hole?  Would the ``warping of
space'' result in {\em visibly} curved thread?

Of course, no real string is perfectly weightless or inextensible.
With tension~$T$, gravitational acceleration $g$ and (proper) mass per
(proper) unit length~$\mu$, then a string will behave as though light
on lengthscales $L\lesssim T/\left(\mu g\right)$.  Using
$c^4/\left(4GM\right)$ for the strength of gravity at the event
horizon of a black hole with mass $M$ gives $L\lesssim TM/\left(\mu
c^4\right)$.  For the string to behave as though light on lengthscales
comparable with the Schwarzschild radius would require $T\gtrsim\mu
c^2$.  Exceeding this criterion would violate the dominant energy
condition (and in this context we note that transverse waves would
propagate at speed~$\sqrt{T/\mu}\sim c$).  Nevertheless, the
idealization of a string as light and inextensible is an interesting
and informative limit: Rindler~\cite{rindler}, for example, considers
lowering a heavy object to the event horizon on the end of such a
string to illustrate energetic arguments in GR.

Dynamic definitions of straight line also run into difficulties:
counterintuitive behaviour such as inward-directed centrifugal force
in the vicinity of a black hole is well
known~\cite{abramowicz1992,abramowicz1993}.


The Schwarzschild metric

\begin{equation}\label{schwarzschild}
ds^2= -dt^2\left(1-2M/r\right) +\frac{dr^2}{1-2M/r} + r^2\left(d\theta^2 + \sin^2\theta d\phi^2\right),
\end{equation}

\noindent where units in which~$G=c=1$ are used, has been known for
over a century~\cite{schwarzschild1916}.  Here I consider the spatial
component~$ds^2= dr^2/\left(1-2M/r\right) + r^2d\phi^2$, confined to
equatorial points~($\theta=\pi/2$); we understand~$r\geqslant 2M$
throughout.  A circular string at constant~$r$ will, trivially, have
proper length~$2\pi r$.

Considering first the radial component we may calculate the proper
length $L\left(r_1,r_2\right)$ of a string stretching
from~$r=r_1\geqslant 2M$ to~$r_2\geqslant r_1$ at constant
$\theta,\phi$:

% \begin{equation}\label{radial_string_length_properunits}
%   \int_{r=r_1}^{r_2}\frac{dr}{\sqrt{1-1/r}}=
%   \left.
%   \sqrt{r(r-1)} +\frac{1}{2}\log\left(
%   \frac{\sqrt{r}+\sqrt{r-1}}{\sqrt{r}-\sqrt{r-1}}\right)
%   \right|_{r_1}^{r_2}
%   \end{equation}

\begin{equation}\label{radial_string_length}
  \int_{r=r_1}^{r_2}\frac{dr}{\sqrt{1-2M/r}}=
  \left.
  \sqrt{r(r-2M)} +\frac{1}{2}\log\left(
  \frac{\sqrt{r}+\sqrt{r-2M}}{\sqrt{r}-\sqrt{r-2M}}\right)
  \right|_{r_1}^{r_2}
  \end{equation}

Observe that integration may be performed with a lower limit
of~$r_1=2M$ with no difficulty, even though the radial component of the
Schwarzschild metric approaches infinity there.

Elementary methods show that the proper length $\ell$ of radial
strings has some non-Euclidean features.  Firstly, consider the proper
length from $r=2M$ to $r=2M(1+w)$ where $w>0, \left|w\right|\ll 1$.
Then

%  \begin{equation}\label{ell_properunits}
%    \ell=\int_{r=1}^{r=1+w}ds
%    =2\sqrt{w} + \frac{1}{3}w^{3/2} + O(w^{5/2})
%  \end{equation}

 \begin{equation}\label{ell}
   \ell =
   \int_{r=2M}^{r=2M(1+w)}ds
   =2M\left(2\sqrt{w} + \frac{1}{3}w^{3/2} + O(w^{5/2})\right).
 \end{equation}

This would indicate that, close to the Schwarzschild radius $r=2M$,
large (but finite) amounts of string are required to connect two
points with slightly differing radial coordinates.  Passing to the
case $w\gg 1$ we have an asymptotic expansion that begins

% \begin{equation}\label{asymptotic_ell_properunits}
%   \ell = w +  \frac{1}{2}\log w + \frac{1}{2} + \log(2)  +  \frac{1}{8}w^{-1} + O(w^{-2}),
% \end{equation}
% 
\begin{equation}\label{asymptotic_ell}
  \ell = 2M\left(w +  \frac{1}{2}\log w + \frac{1}{2} + \log(2)  +  \frac{1}{8}w^{-1} + O(w^{-2})\right),
\end{equation}

\noindent suggesting that there is ``extra radial space'', of length
over $1.19\cdot 2M$, sensible at large distances from the event
horizon.  Radial strings have proper length longer than the
Schwarzschild radial coordinate difference $w$, a nice illustration of
the failure of Euclidean geometry.

\section{Nonradial string}
Integrating along a path between two
points~$p_1=\left(r_1,\phi_1\right)$ and~$p_2=\left(r_2,\phi_2\right)$
gives the {\em proper} length of the path, which is the length of an
inextensible string joining~$p_1$ and~$p_2$.  A taut but light string
from~$p_1$ to~$p_2$ will adopt an extremal-length path.  Such paths
may be found by the calculus of variations; parametrizing a curve in
terms of~$r=r\left(\phi\right)$ gives us a path length of

% \begin{equation}
% %  \int_{p_1}^{p_2}\sqrt{\frac{\left(r'\right)^2}{1-2M/r} + r^2}d\phi=
%   \int_{p_1}^{p_2}\left(\frac{\left(r'\right)^2}{1-1/r} + r^2\right)^\frac{1}{2}d\phi=
%   \int_{p_1}^{p_2}F\left(r,r'\right)d\phi
% \end{equation}

\begin{equation}
%  \int_{p_1}^{p_2}\sqrt{\frac{\left(r'\right)^2}{1-2M/r} + r^2}d\phi=
  \int_{p_1}^{p_2}\left(\frac{\left(r'\right)^2}{1-2M/r} + r^2\right)^\frac{1}{2}d\phi=
  \int_{p_1}^{p_2}F\left(r,r'\right)d\phi
\end{equation}

\noindent where dashes denote differentiation with respect to~$\phi$.  The
Euler-Lagrange equations for this system,
$\frac{d}{d\phi}\frac{\partial F}{\partial r'}-\frac{\partial
  F}{\partial r}=0$ give, after simplification, an expression for the
second derivative of~$r$:

% \begin{equation}\label{rdashdash_properunits}
%   r''\left(\phi\right) =
%   (r-1) + \frac{(4r-3)\left(r'\right)^2}{2r\left(r-1\right)},\qquad r>1.
% \end{equation}
\begin{equation}\label{rdashdash}
  r''\left(\phi\right) =
  (r-2M) + \frac{(4r-6M)\left(r'\right)^2}{2r\left(r-2M\right)},\qquad r>2M.
\end{equation}

\noindent 
Solutions to equation~\ref{rdashdash} correspond to a taut string
confined to Flamm's paraboloid~\cite{flamm1916}; see
Figure~\ref{flamm}.  It is interesting to compare
equation~\ref{rdashdash} with the corresponding equation for null
geodesics, which is usually given~\cite{wald} in terms of $u=2M/r$ as
$u''=3u^2/2-u$, but is equivalent
to~$r''\left(\phi\right)=r-3M+2M\left(r'\right)^2/r$.  This shows that
light, being sensitive to the time component of the metric, behaves
differently from string, which is not.

There does not appear to be a simple analytical solution to
equation~\ref{rdashdash} but nevertheless we may make several
observations.  Equation~\ref{rdashdash} is a second-order nonlinear
ordinary differential equation and thus has two constants of
integration.  There are two qualitatively different types of
solutions: those with a tangential point ($r'=0,r>2M$), and those
without.  Consider first solutions with such a point: there exists a
pair~$\left(r,\phi\right), r>2M$ at which~$r'(\phi)=0$.  This point
must be unique, as~$r''(\phi)>0$, and be the closest approach of the
string to the event horizon.

Further, because the system is symmetrical
under~$\phi\longrightarrow-\phi$, the string must be symmetrical about
at this point.  The (signed) radius of curvature~$R$ is readily
evaluated:

% \begin{equation}\label{roc_properunits}
%   R = \frac{
%     \left(r^2 + \left(r'\right)^2\right)^\frac{3}{2}
%   }{
% %    \left|
%     r^2 + 2\left(r'\right)^2-rr''
% %    \right|
%   }
%   =
%   \frac{
%     \left(r^2 + \left(r'\right)^2\right)^\frac{3}{2}
%   }{
%     r-\frac{\left(r'\right)^2}{2\left(r-1\right)}
%   }\qquad\mbox{along a taut string}
% \end{equation}

\begin{equation}\label{roc}
  R = \frac{
    \left(r^2 + \left(r'\right)^2\right)^\frac{3}{2}
  }{
%    \left|
    r^2 + 2\left(r'\right)^2-rr''
%    \right|
  }
  =
  \frac{
    \left(r^2 + \left(r'\right)^2\right)^\frac{3}{2}
  }{
    r-\frac{\left(r'\right)^2}{2\left(r-2M\right)}
  }\qquad\mbox{along a taut string}
\end{equation}

\noindent which would give~$R$ as a function of~$\left(r,r'\right)$.
Considering~$r'(\phi)=0$ shows that if a string is tangential at a
point with~$r_0>2M$, then it has a radius of curvature of~$r_0^2/2M$
at that point.  Any circular path, including a circular orbit
at~$r=r_0$, has a radius of curvature of $r_0<r_0^2/2M$, showing that
circular orbits cannot follow a taut string at closest approach (as in
the classical case).  Also observe that null geodesics have a radius
of curvature of $r_0^2/3M < r_0^2/2M$, showing that light cannot
follow a taut string at closest approach either.  The string has a
point of inflection if~$\left(r'\right)^2=2r(r-2M)$.

Compare the above analysis with a taut {\em heavy} string which
adopts, at least locally, a catenary configuration at closest
approach~\cite{nguyen2007}.  A heavy string is manifestly convex to
the singularity in the sense that $R<0$ if $\phi'=0$, in contrast to
light strings which have $R>0$.

If, conversely, $r'(\phi)\neq 0$ whenever~$r>2M$, then the string has
only one free end; the other is tangential to the event horizon (in
the sense that ${\displaystyle \lim_{r\longrightarrow 2M^+}r'=0}$),
except in the degenerate case of constant~$\phi$ when the string is
radial for~$r>2M$.

\section{Numerical results}

Here I solve equation~\ref{rdashdash} using the \verb+deSolve+ package
for ordinary differential equations~\cite{soetart2010}.  Non-radial
solutions fall into two distinct classes: those with $r>2M$ at all
points, and those which attain $r=2M$, in which case the string is
tangential to the event horizon.

We first consider strings for which $r>2M$ everywhere (that is,
completely exterior strings).
Figure~\ref{closest_approach_non_self_intersecting} shows a sequence
of non-self-intersecting numerical solutions to
equation~\ref{rdashdash}, rotated so that the tangential point of
closest approach occurs at~$\phi=0$.
Figure~\ref{closest_approach_self_intersecting} shows the
corresponding diagram for self-interesecting strings
(Figure~\ref{light_closest_approach} shows equivalent null geodesics
for comparison).  A bisection technique reveals that if the closest
approach is greater than a critical value of about~$\simeq 1.0761\cdot
2M$, the string does not self-intersect; if the closest approach is
less than this value, then the string crosses itself at~$\phi=\pi/2$.
As discussed above, if $\left(r'\right)^2=2r(r-2M)$, the string has a
point of inflection at which point the radius of curvature becomes
infinite.  Figure~\ref{curvature_switch} shows non-self intersecting
strings rotated so that the point of inflection occurs at~$\phi=0$; we
see that each string has a finite length in which it is convex to the
origin and two semi-infinite stretches in which the curvature is
negative.

There are other ways to parametrize completely exterior strings;
Figure \ref{strings_r_equals_2}, for example, shows strings passing
through~$(4M,0)$.  Figure~\ref{light_r_equals_2} shows light paths for
comparison.

The cases where the string attains $r=2M$ are tangent to the event
horizon (or, exceptionally, purely radial).
Figure~\ref{fixed_EH_intersection} shows strings tangential to the
event horizon at~$\phi=0$ and Figure~\ref{one_free_end_r_equals_2}
shows tangential strings passing through $(4M,0)$.

\section{Discussion}

Light inextensible strings under tension adopt non-trivial
configurations including self-intersecting paths.  The path of such
strings interprets the concept ``straight line'' in general
relativity, others being null geodesics and world lines of massive
objects.  Comparing a world line with that of nearby taut strings thus
furnishes another insight into path curvature, complementing the
analysis of Abramowicz~\cite{abramowicz1992}.

In Figure~\ref{closest_approach_non_self_intersecting}, one's
classical intuition, and d'Alembert's principle, would suggest that
the black hole should accelerate to the left.  Note that the strings
are already ``straight'' in sense that they are minimal length; so
altering the tension would not change their path.  However, it is not
clear by what mechanism the required force would be transmitted: the
situation is steady, so no gravitational waves are produced, and the
string is inextensible so stores no energy.  Further work might
include analysis of taut strings near a spinning black hole.

%%Vancouver style references.

\section*{Supplementary Material}

Calculations are performed using the R programming language~\cite{rcore2018};
source code for the figures is available at
\\
\\
{\tt https://github.com/RobinHankin/string.git}


\section*{Figures}

\begin{figure}[h!] % Created by "flamm.R"
\centering
\includegraphics[width=170mm]{flamm_string.pdf}
\caption{Perspective view of Flamm's paraboloid with superimposed minimal-length
  path corresponding to a taut, light string}
\label{flamm}
\end{figure}

\begin{figure}[p] % Created by "closest_approach.R"
\centering
\includegraphics[width=170mm]{closest_approach_nonselfintersecting.pdf}
\caption{Non-self intersecting strings arranged by distance~$d$ of
  closest approach to the event horizon, occurring tangentially
  at~$\phi=0$}
\label{closest_approach_non_self_intersecting}
\end{figure}

\begin{figure}[p] % created by "closest_approach2.R"
\centering
\includegraphics[width=170mm]{closest_approach_selfintersecting.pdf}
\caption{Self-intersecting strings shown for $\phi\geqslant 0$.  Closest
  approach to the event horizon occurs at $\phi=0$ (3 o'clock) and
  $r/2M\in\left(1,1.0761\right)$; strings are symmetrical about $\phi=0$
  but for clarity only $\phi\geqslant< 0$ is shown}
\label{closest_approach_self_intersecting}
\end{figure}

\begin{figure}[p] % Created by "light_closest_approach.R"
\centering
\includegraphics[width=170mm]{light_closest_approach.pdf}
\caption{Null geodesics in the Schwarzschild geometry, tangential
  at~$\phi=0$.  Note the differences between these curves and the taut
  strings shown elsewhere: unlike taut strings, null geodesics have
  positive curvature everywhere, may cross the event horizon inwards,
  and are never tangential to the event horizon}
\label{light_closest_approach}
\end{figure}

\begin{figure}[p]
\centering
\includegraphics{curvature_switch.pdf} % created by "radius_of_curvature_switch.R"
\caption{Strings arranged by point at which the radius of curvature
  switches, which occurs at~$\phi=0$.  Colour changes from black to
  blue at points of infinite~$R$; black indicates convex to the origin
  and blue indicates concave regions}
\label{curvature_switch}
\end{figure}

\begin{figure}[p]
\centering
\includegraphics{angle_at_r_equals_2.pdf} % created by "angle_at_r_equals_2.R"
\caption{Strings passing through $(4M,0)$}
\label{strings_r_equals_2}
\end{figure}

\begin{figure}[p]
\centering
\includegraphics{light_start_at_r_equals_2.pdf} % created by "angle_start_at_r_equals_2.R"
\caption{Null geodesics passing through $(4M,0)$, shown for
  $0\leqslant\phi\leqslant 2\pi$}
\label{light_r_equals_2}
\end{figure}

\begin{figure}[p] % created by "one_free_end_fixed_EH_intersection.R"
\centering
\includegraphics[width=170mm]{one_free_end_fixed_EH_intersection.pdf}
\caption{Strings tangential to event horizon at $(1,0)$}
\label{fixed_EH_intersection}
\end{figure}

\begin{figure}[p] % created by "one_free_end_r_equals_2.R"
\centering
\includegraphics[width=170mm]{one_free_end_r_equals_2.pdf}
\caption{Strings tangential to the event horizon passing through $(4M,0)$}
\label{one_free_end_r_equals_2}
\end{figure}



 \section*{Conflict of interest}
 The author declares that he has no conflict of interest.

\subsection*{Supplementary material}

R code for all statistical analysis given here is available at

{\tt https://github.com/RobinHankin/hyper2}.

\section*{References}

\bibliography{stringrefs.bib}

\end{document}

