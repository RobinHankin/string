\documentclass[12pt]{article}
\usepackage{xcolor}
\begin{document}

I appreciate the author responding to each of my previous questions
and suggestions. For the most part I am satisfied with the author’s
responses and subsequent modifications to the manuscript. While it is
not my purpose to annoy the author or delay an editorial decision on
publication, a small number of comments about minor points remain:

1. The discussion about circumference and $2\pi r$ may be, to the
undergraduate reader to whom the Schwarzschild metric is still quite
new, a source of confusion regarding the logic behind the metric. In
the revised manuscript we read of ``constant $r$ whose proper length
is given by integration," where the integral is

\[
\int_{0}^{2\pi} r\,d\phi = 2\pi r
\]


While the integration is certainly true, it this context it appears to
express a tautology because the $r$- coordinate is defined as the
circumference of a shell (centered on the origin) divided by $2\pi$.
The passage just quoted makes it sound (at least to me) like
``circumference = $2\pi r$ is a derived result, when in fact it the
logic goes the other way---the measured circumference provides the
definition of $r$ (see Taylor \& Wheeler, Black Holes, page 2-7, which
in T\&W pagination means page 7 in Ch. 2; T\&W is a popular
entry-level textbook for undergraduates interested in general
relativity).

2. The author's explanation registers with more clarity now in the
passage about string being ``arbitrarily" large.  At first reading
this sounds like the author says that the length of a tight string
fixed between two points may be arbitrary, notwithstanding the
distinction between ``arbitrary" and ``arbitrarily."  Here is what I
now understand the author to be saying about the value of l in flat
and in Schwarzschild spacetimes:

\[
\ell_\mathrm{flat} = 2Mw
\]

and, asymptotically,

\[
\ell_\mathrm{Sch}\sim 2Mw + \log w
\]

and thus their difference is


\[
\ell_\mathrm{Sch} - \ell_\mathrm{flat}\sim M\log w
\]

which goes to infinity logarithmically; in other words, for an
infinite string the Schwarzschild metric is a ``worse infinity" which
diverges faster than what occurs in flat spacetime.  So far, so good.
I guess what bothers me is the statement that ``even far from [emphasis
  added] the event horizon, radial strings are arbitrarily longer than
flat space would imply...".  So long as $w$ is large (however far from
the event horizon) but finite, the lengths in both metrics are precise
numbers, not arbitrary.  Only when $w$ goes to infinity does the
length in Schwarzschild spacetime become infinitely longer than the
length in flat spacetime, although though both are infinite.  I
suppose this passage would sit better with this reader by merely
stating that the asymptotic difference between $\ell_\mathrm{Sch}$ and
$\ell_\mathrm{flat}$ diverges logarithmically.  But perhaps this is
only a matter of taste.

3. I still must disagree with the author about the coefficient of the
logarithm term in Eq. (2).  The original manuscript put this
coefficient at $1/2$; I argued that it is $2M$ because of the
corresponding equation in T\&W (their Eq. [27] on p. 2-28).  I also
find disagreement with Eq. (2) on a sign in the denominator of the log
term.  Standing ready to be corrected, please allow me to show why
this reader thinks these discrepancies exist.  We are to evaluate

\[
\ell=\int_{r_1}^{r_2}\frac{dr}{\sqrt{1-2M/r}} =
\int_{r_1}^{r_2}\frac{\sqrt{r}\,dr}{\sqrt{r-2M}},
\]

Let $r=z^2$ so that $dr=2z\,dz$, and let $2M=a^2$.  Then 

\[
\ell=2\int_{r_1}^{r_2} \frac{z^2\,dz}{\sqrt{z^2-a^2}}.
\]

 Using the integral table in the CRC handbook, integral No. 171,

\[
\int\frac{dx}{\sqrt{x^2\pm a^2}}=\frac{x}{2} \sqrt{z^2\pm a^2}
\mp \frac{a^2}{2}\log\left| z+\sqrt{x^2\pm a^2}\right|
\]


 and with the minus sign under the radical in the integrand, I obtain

\[
\ell = 2\left[
  \frac{z}{2}\sqrt{z^2-a^2} + \frac{a^2}{2}\log\left|z+\sqrt{z^2-a^2}\right|
  \right]_{z_1}^{z_2}
\]

\[
=
\sqrt{r_2(r_2-2M)}-\sqrt{r_1(r_1-2M)} + 2M\log\left|
\frac{\sqrt{r_2}+\sqrt{r_2-2M}}{\sqrt{r_1}+\sqrt{r_1-2M}}\right|
\]

In contrast, Eq (2) reads (points of disagreement in red):


\[
\sqrt{r_2(r_2-2M)}-\sqrt{r_1(r_1-2M)} 
+\color{red}M
\color{black}\log\left|
\frac{\sqrt{r_2}+\sqrt{r_2-2M}}{\sqrt{r_1}
  \color{red}-
  \color{black}\sqrt{r_1-2M}}\right|
\]


Other than these minor points, I have no further concerns about the
paper. I find the project very interesting, offering much to think
about. Having worked with undergraduate physics majors who wish to
learn some general relativity, once such students are made familiar
with the Schwarzschild metric, this exercise with a tight string
offers a useful tool. I leave it to the editors and authors to decide
what to do with these my latest suggestions regarding the revised
manuscript. I thank the author for taking my previous suggestions and
concerns seriously.

\end{document}
