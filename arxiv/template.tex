\documentclass{article}

\usepackage{arxiv}

\usepackage[utf8]{inputenc} % allow utf-8 input
\usepackage[T1]{fontenc}    % use 8-bit T1 fonts
\usepackage{hyperref}       % hyperlinks
\usepackage{url}            % simple URL typesetting
\usepackage{booktabs}       % professional-quality tables
\usepackage{amsfonts}       % blackboard math symbols
\usepackage{nicefrac}       % compact symbols for 1/2, etc.
\usepackage{microtype}      % microtypography
\usepackage{cleveref}       % smart cross-referencing
\usepackage{lipsum}         % Can be removed after putting your text content
\usepackage{graphicx}
\usepackage{natbib}
\usepackage{doi}
\title{Proper radial distance as a coordinate in the Schwarzschild metric}

% Here you can change the date presented in the paper title
%\date{September 9, 1985}
% Or remove it
%\date{}

\author{ \href{https://orcid.org/0000-0001-5982-0415}{\includegraphics[scale=0.06]{orcid.pdf}\hspace{1mm}Robin K. S. Hankin}\thanks{
\url{https://academics.aut.ac.nz/robin.hankin}}\\
              AUT University\\
              Auckland, New Zealand\\
	\texttt{hankin.robin@gmail.com} \\
}

% Uncomment to override  the `A preprint' in the header
%\renewcommand{\headeright}{Technical Report}
%\renewcommand{\undertitle}{Technical Report}
%\renewcommand{\shorttitle}{Proper radial distance}

%%% Add PDF metadata to help others organize their library
%%% Once the PDF is generated, you can check the metadata with
%%% $ pdfinfo template.pdf
\hypersetup{
pdftitle={Proper radial distance as a coordinate in the Schwarzschild metric},
pdfsubject={q-bio.NC, q-bio.QM},
pdfauthor={Robin K. S.~Hankin},
pdfkeywords={Black holes, Schwarzschild metric, proper distance}
}
\begin{document}
\maketitle

\begin{abstract}
The Schwarzschild solution to the spherically symmetric vacuum
Einstein field equations is usually presented in terms of
Schwarzschild coordinates $(t,r,\phi,\theta)$.  However, we are free
to consider reparametrizations in terms of functions of the
Schwarzschild coordinates.  Here I consider the Schwarzschild solution
in terms of $(t,u_1\phi,\theta)$, where $u_1=u_1(r)$ is the
straight-line proper length from a point to the event horizon.  This
coordinate system has some interesting properties and affords new
physical insight to black holes.  I also consider related
parameterizations $u_2$ and $u_3$ which preserve proper area or volume
instead of proper length.
\end{abstract}


% keywords can be removed
\keywords{First keyword \and Second keyword \and More}


\section{Introduction}
The Schwarzschild metric is usually written

\begin{equation}
  ds^2=-dt^2\left(1-\frac{2GM}{c^2r}\right)
<  +\frac{dr^2}{\left(1-\frac{2GM}{c^2r}\right)} + r^2(\sin^2\theta
  d\phi^2 + d\theta^2)
\end{equation}

where $G$ is the gravitational constant, $c$ the speed of light, and
$M$ the mass of the black hole.  However, it is much better to use
units in which $c=1$ and $2MG/c^2=1$ [that is, the speed of light is
  unity and the Schwarzschild radius is unity].  For a solar mass
black hole ($2\times 10^{30}\,\mathrm{kg}$) we have the unit of
distance being about $3\,\mathrm{km}$ and the unit of time about
$10\,\mathrm{ms}$.  With such units the Schwarzschild metric becomes

\begin{equation}
  ds^2=-dt^2\left(1-1/r\right)
  +\frac{dr^2}{\left(1-1/r\right)} + r^2(\sin^2\theta d\phi^2 + d\theta^2)
\end{equation}

If we consider constant $\phi,\theta$ then we have 

\label{sec:headings}

\begin{equation}
  ds^2=-dt^2\left(1-1/r\right)
  +\frac{dr^2}{\left(1-1/r\right)}.
\end{equation}

The proper distance from a point at Schwarzschild coordinate $r$ would be

\begin{equation}
  u_1(r)=
  \int_{r'=1}^r\frac{dr'}{\sqrt{1-1/r'}}=
  \sqrt{r(r-1)} + \frac{1}{2}\log\frac{\sqrt{r}+\sqrt{r-1}}{\sqrt{r}-\sqrt{r-1}}
  \end{equation}

\cite{hankin2021} shows that

\citep{hankin2021} shows that

\citet{hankin2021} shows that

\cite{hankin2020}


\begin{equation}
  u_1(1+w)=2\sqrt{w} + \frac{1}{3}w^{3/2} + O(w^{7/2})
\end{equation}
for $w\ll 1$, and 

\begin{equation}
  u_1(1+w)=w + \frac{1}{2}\log w + \frac{1}{2} +
  \log(2) + \frac{1}{8}w^{-1} + O(w^{-2})
\end{equation}

for $w\gg 1$.


\bibliographystyle{unsrtnat}
\bibliography{references}  


\end{document}
