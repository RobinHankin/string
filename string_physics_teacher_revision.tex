\documentclass[review]{elsarticle}

\usepackage{lineno,hyperref}
\usepackage{amsmath}
\usepackage{amssymb}
\usepackage{booktabs}
\usepackage{units}
\usepackage{xcolor}
\usepackage[superscript,biblabel]{cite}

\modulolinenumbers[5]

\journal{The Physics Teacher}

\bibliographystyle{unsrt}

\begin{document}
\begin{frontmatter}
\title{Light inextensible strings (threads) under tension in the
  Schwarzschild geometry: a new interpretation for ``straight line''
  in general relativity}




% \PACS{PACS code1 \and PACS code2 \and more}
% \subclass{MSC code1 \and MSC code2 \and more}


\begin{keyword}
Black holes\sep light inextensible string\sep minimal-length path
\end{keyword}

\end{frontmatter}

\linenumbers
\section{Introduction}

Light inextensible string (thread) under tension is a stalwart feature
of elementary physics.  Here I show how considering such string in the
vicinity of a black hole, with the help of computer algebra systems,
can generate insight into the Schwarzschild geometry in the context of
an undergraduate homework problem, specifically by re-interpreting the
concept ``straight line''.  Other definitions of ``straight'' might
include null geodesic paths and worldlines of objects in free fall;
light taut string furnishes a third distinct interpretation.  The path
itself is given by straightforward numerical solution to the
Euler-Lagrange equations.  At closest approach to the event horizon,
the string is tangential; at this point the string is {\em visibly}
curved, surely a memorable and informative insight.  The geometry of
the Schwarzschild metric induces some interesting nonlocal phenomena:
if the distance of closest approach is less than about $1.0761\cdot
2M$, the string self-intersects, even though it is everywhere under
tension.  

General relativity is a topic of long-standing fascination for physics
undergraduates~\cite{christensen2012}.  Students appreciate everyday
illustrations of the concepts involved, as recent educational work on
gravitational wave visualisations attests~\cite{overduin2018}.  Here I
show how considering light inextensible string can be used to
interpret the concept of ``straight line'' in the vicinity of a black
hole.  The analysis presented here would be suitable for an
undergraduate homework problem~\cite{romano2019}.

A light string's being under tension means that its path will be an
extremum of proper length.  In flat space, the calculus of variations
proves that the path is straight in the sense of zero curvature.  In
terrestrial gravity, standard $25\times
10^{-6}\unit{kg}\cdot\unit{m}^{-1}$ cotton sewing thread at its breaking
strain of about $40\unit{N}$ would have a sag of~${\sim} 6\times
10^{-6}\unit{m}$ over a $1\unit{m}$ horizontal scale: barely perceptible.
How does such a quotidian observation translate into the vicinity of a
black hole?  Would the ``warping of space'' result in {\em visibly}
curved thread?

Of course, no real string is perfectly weightless or inextensible.
Students could use dimensional analysis to assess the accuracy of such
approximations.  With tension~$T$, gravitational acceleration $g$ and
(proper) mass per (proper) unit length~$\mu$, then a string will
behave as though light on lengthscales $L\lesssim T/\left(\mu
g\right)$.  Using $c^4/\left(4GM\right)$ for the strength of gravity
at the event horizon of a black hole with mass $M$ gives $L\lesssim
TM/\left(\mu c^4\right)$.  For the string to behave as though light on
lengthscales comparable with the Schwarzschild radius would require
$T\gtrsim\mu c^2$.  Exceeding this criterion would violate the
dominant energy condition (and in this context we might remind
students that transverse waves would propagate at
speed~$\sqrt{T/\mu}\sim c$).  Nevertheless, the idealization of a
string as light and inextensible is an interesting and instructive
limit: Rindler~\cite{rindler}, for example, considers lowering a heavy
object to the event horizon on the end of such a string to illustrate
energetic arguments in GR.

\section{The Schwarzschild metric}

The vacuum Einstein field equations for a static, spherically
symmetric mass are solved by the Schwarzschild
metric~\cite{schwarzschild1916}.  In geometrized units, that is, a
system in which~$G=c=1$, the Schwarzschild metric is:

\begin{equation}\label{schwarzschild}
ds^2= -dt^2\left(1-2M/r\right) +\frac{dr^2}{1-2M/r} + r^2\left(d\theta^2 + \sin^2\theta d\phi^2\right).
\end{equation}

Many undergraduate courses will present the Schwarzschild metric {\em
  ex nihilo}, in which case its pedagogical value would be to
illustrate non-Newtonian features of GR such as time dilation.  The
most striking of these are temporal: suppose pulses of light are
emitted from a distant source [such as a pulsar] with interval $t_P$;
the coefficient of $dt^2$ implies that a student, stationary at
distance $r>2M$ from the singularity will perceive these pulses to
have an interval of~$t_P/\sqrt{1-2M/r} < t_P$ [it might be worth
  emphasising that coordinate ``$t$'' coincides with Lorentzian time
  for an observer at infinity].  Further, if our student emits pulses
at (local) interval $t_L$, then the distant observer would see these
pulses with an interval of $t_L\sqrt{1-2M/r}$, and this approaches
infinity as $r\longrightarrow 2M$ and the distant observer sees the
student ``freezing''.  Nevertheless, the free-fall time from the Event
Horizon to the singularity is finite.

We also see a failure of classical geometry: the coordinate $r$ has
the defining feature that the circumference of a circle of radius $r$
has circumference $2\pi r$; but radial distances are not Euclidean.
This is frequently expressed by saying that space is ``warped'', a
concept that has proved troublesome for generations of students.
Considering taut strings surely furnishes students with a useful
conceptual tool to understand the non-Euclidean geometry near a black
hole.  To investigate this phenomenon, we consider the spatial
component~$ds^2= dr^2/\left(1-2M/r\right) + r^2d\phi^2$ of the
Schwarzschild metric, confined to equatorial points~($\theta=\pi/2$);
we understand~$r\geqslant 2M$ throughout.

Students might need to be reminded of the concepts of proper time and
proper length: these are quantities measured {\em locally}.  In
particular, we note that proper length of an object is the length as
measured by a local observer at rest with respect to the object.  The
proper length of a string is the length measured by an inchworm as it
crawls along the string.  A circular string at constant~$r$ will,
trivially, have proper length~$2\pi r$ [this follows from considering
  the Schwarzschild metric at constant $t,r,\theta$; the string will
  then have proper (inchworm) length $\int_{\phi=0}^{2\pi}ds=2\pi r$].
Considering now the radial component we may calculate the proper
length $L\left(r_1,r_2\right)$ of a string stretching
from~$r=r_1\geqslant 2M$ to~$r_2\geqslant r_1$ at constant
$\theta,\phi$:

\begin{equation}\label{radial_string_length}
  \int_{r=r_1}^{r_2}\frac{dr}{\sqrt{1-2M/r}}=
  \left.
  \sqrt{r(r-2M)} +\frac{1}{2}\log\left(
  \frac{\sqrt{r}+\sqrt{r-2M}}{\sqrt{r}-\sqrt{r-2M}}\right)
  \right|_{r_1}^{r_2}
  \end{equation}

(students might need to be told explicitly that the right hand side of
equation~\ref{radial_string_length} not being equal to its Euclidean
value of $r_2-r_1$ is a direct illustration of the non-flatness of
space near a black hole).  Observe that the integration may be
performed with a lower limit of~$r_1=2M$ with no difficulty, even
though the radial component of the Schwarzschild metric approaches
infinity there.

One criticism of this analysis from a pedagogical perspective is that
the algebra is involved and nonintuitive.  However, we may apply
inchworm thinking via calculus and asymptotic series to show
simplified expressions for the proper length $\ell$ of radial strings.
Firstly, consider the proper length from $r=2M$ to $r=2M(1+w)$ where
$w>0, \left|w\right|\ll 1$.  Then

 \begin{equation}\label{ell}
   \ell =
   \int_{r=2M}^{r=2M(1+w)}ds
   =2M\left(2\sqrt{w} + \frac{1}{3}w^{3/2} + O(w^{5/2})\right).
 \end{equation}

(compare the flat space string length of $2Mw$).  This would indicate
 that, close to the Schwarzschild radius $r=2M$, large (but finite)
 amounts of string are required to connect two points with slightly
 differing radial coordinates.  Passing to the case $w\gg 1$ we have
 an asymptotic expansion that begins


\begin{equation}\label{asymptotic_ell}
  \ell = 2M\left(w +  \frac{1}{2}\log w + \frac{1}{2} + \log(2)  +  \frac{1}{8}w^{-1} + O(w^{-2})\right),
\end{equation}

\noindent suggesting that there is ``extra space'', sensible at large
distances from the event horizon, with radial strings being
arbitrarily longer than flat space would allow: a nice illustration of
the failure of Euclidean geometry.

\section{Three types of straight line}

There are at least three types of ``straight line'' to consider when
studying general relativity.  In the context of
equation~\ref{schwarzschild} we have:

\begin{enumerate}
\item Null geodesics: paths for which $ds^2=0$ at all points.  Light
  travels on null geodesics and is arguably the ``straightest'' of
  straight lines.
\item Timelike geodesics: paths for which $ds^2>0$ everywhere, and for
  which $\int\sqrt{ds^2}$ is a minimum.  These are the paths that
  objects in freefall follow.
\item Taut string: paths for which $dt=0$, thereby considering the
  configuration of a string.  If the proper length
  $\int\sqrt{dr^2+r^2d\theta^2}$ is a minimum, the string is taut in
  the sense that its inchworm-measured length is a minimum.
\end{enumerate}

The path of a taut string seems to be a natural and intuitive
interpretation of straight line, with immediate everyday
interpretation, yet has been neglected in the pedagogical literature.
Nontrivial taut strings require analysis of nonradial paths, which I
consider below.

\section{Nonradial string}
Integrating along a path between two
points~$p_1=\left(r_1,\phi_1\right)$ and~$p_2=\left(r_2,\phi_2\right)$
gives the {\em proper} length of the path, which is the length of an
inextensible string joining~$p_1$ and~$p_2$.  A taut but light string
from~$p_1$ to~$p_2$ will adopt an extremal-length path.  Such paths
may be found by the calculus of variations; parametrizing a curve in
terms of~$r=r\left(\phi\right)$ gives us a path length of


\begin{equation}
%  \int_{p_1}^{p_2}\sqrt{\frac{\left(r'\right)^2}{1-2M/r} + r^2}d\phi=
  \int_{p_1}^{p_2}\left(\frac{\left(r'\right)^2}{1-2M/r} + r^2\right)^\frac{1}{2}d\phi=
  \int_{p_1}^{p_2}F\left(r,r'\right)d\phi
\end{equation}

\noindent where dashes denote differentiation with respect to~$\phi$.
The Euler-Lagrange equations for this system,
$\frac{d}{d\phi}\frac{\partial F}{\partial r'}-\frac{\partial
  F}{\partial r}=0$ give an expression for the second derivative
of~$r$ (Mathematica idiom would be {\tt EulerEquations[Sqrt[y[x]\^{}2
      + y'[x]\^{}2/(1-1/y[x])],y[x],x]}):

\begin{equation}\label{rdashdash}
  r''\left(\phi\right) =
  (r-2M) + \frac{(2r-3M)\left(r'\right)^2}{r\left(r-2M\right)},\qquad r>2M.
\end{equation}

\noindent Solutions to equation~\ref{rdashdash} correspond to a taut
string confined to Flamm's paraboloid~\cite{flamm1916}; see
Figure~\ref{flamm}.  At this point it might be worth reminding
students that a straight line in flat space,
$r=r_0\sec\left(\phi-\phi_0\right)$, obeys
$r''=r+2\left(r'\right)^2/r$, and that equation~\ref{rdashdash}
reduces to this in the limit $r/2M\longrightarrow\infty$.

It is interesting to compare equation~\ref{rdashdash} with the
corresponding equation for null geodesics, which is usually
given~\cite{wald} in terms of $u=2M/r$ as $u''=3u^2/2-u$, but is
equivalent to~$r''\left(\phi\right)=r-3M+2\left(r'\right)^2/r$.  This
shows that light, being sensitive to the time component of the metric,
behaves differently from string, which is not.


Here we consider only strings exterior to the event horizon, $r>2M$,
which have a well-defined point of closest approach to the
singularity.  Because the system is invariant
under~$\phi\longrightarrow-\phi$, the string must be symmetrical about
this point.  If $\phi'$ is small, then a string follows $r''=r-2M$,
while tangential null geodesics follow $r''=r-3M$.  Thus a taut string
would {\em look} curved to a local observer, having a (visual) radius
of curvature of $r_0^2/M\sqrt{1-2M/r_0}$ (the extra term comes from
the fact that appearance is driven by local $ds$, not Schwarzschild
$r$).  At the photon sphere $r_0=3M$, such a string would appear to
have a radius of~$\sqrt{27}M$.  For a solar mass black hole this would
be about $15.4\unit{km}$, a readily understandable result.  Of course,
real cotton thread would break here, but the deviation of tangential
light from a taut string's path would be about $0.13\unit{mm}$,
certainly detectable by the naked eye.
\section{Numerical results}

There does not appear to be a simple analytical solution to
equation~\ref{rdashdash} and numerical methods are needed.  In an
educational context, we might use Euler's method to illustrate the
concepts and find $\left(r,r'\right)$---that is, a point in a
two-dimensional phase space---as a function of independent variable
$\phi$.  We specify initial conditions $(r_0,r'_0)$ at $\phi_0$ and
repeatedly increment $\phi$ by a small amount $\delta\phi$:

\begin{eqnarray}
\phi&\longrightarrow&\phi+\delta\phi\nonumber\\
r   &\longrightarrow&r+r'\delta\phi\nonumber\\
r'  &\longrightarrow&r' + r''\delta\phi\qquad\mbox{where $r''$ is from equation~\ref{rdashdash}.}
\end{eqnarray}

Of course, Euler's method is numerically unstable and inefficient.
Here we use the R programming language~\cite{rcore2019} and the
\verb+deSolve+ package for ordinary differential
equations~\cite{soetart2010} using the standard \verb+lsoda+ solver.
The equations typically imply
$\displaystyle\lim_{\phi\longrightarrow\pm\phi_C} r(\phi)=\infty$ for
some critical value $\phi_C$ of $\phi$ [observe $\phi_C=\pi/2$ for
  flat space] and this makes numerical analysis challenging.

First we consider strings for which $r>2M$ everywhere (that is,
completely exterior strings).
Figure~\ref{closest_approach_non_self_intersecting} shows a sequence
of non-self-intersecting numerical solutions to
equation~\ref{rdashdash}, rotated so that the tangential point of
closest approach occurs at~$\phi=0$.

Further analysis reveals something unexpected: for some values of
initial conditions, the string self-intersects even though it is under
tension everywhere and cannot support any nonzero bending moment.  One
might imagine our inchworm confidently declaring that the string was
everywhere ``straight'' in the sense of being a minimal proper length
path, and finding to his astonishment that he has traversed a closed
loop.

Figure~\ref{closest_approach_self_intersecting} shows some
self-interesecting strings (Figure~\ref{light_closest_approach} shows
equivalent null geodesics for comparison).  A bisection technique
reveals that if the closest approach is greater than a critical value
of about~$\simeq 1.0761\cdot 2M$, the string does not self-intersect;
if the closest approach is less than this value, then the string
crosses itself at~$\phi=\pi/2$.

There are other ways to parametrize completely exterior strings;
Figure \ref{strings_r_equals_2}, for example, shows strings passing
through~$(4M,0)$.  Figure~\ref{light_r_equals_2} shows light paths for
comparison.

\section{Discussion}

The concept of ``straight line'' is problematic in general relativity.
Although photon paths provide a definition of straightness, their
world lines are sensitive to the time component of the metric.  Light
taut strings respond only to the spatial components of the metric, and
their configuration is arguably a more intuitive and helpful way to
think about straightness.  The device of using an inchworm to measure
proper distance along a non-extensible string serves to aid
understanding of the non-Euclidean nature of space near a black hole.

Using numerical methods to compute string configuration is suitable
for an undergraduate homework problem; the difficult part is
specifying sensible initial conditions.  The code producing the
figures is available at github,
\url{http://github.com/RobinHankin/string}.

\bibliography{stringrefs.bib}

\clearpage

\section*{Figures}

\begin{figure}[h!] % Created by "flamm.R"
\centering
\includegraphics[width=\linewidth]{flamm_string.pdf}
\caption{Perspective view of Flamm's paraboloid with superimposed
  minimal-length path corresponding to a taut, light string.  Proper
  length along this string corresponds to the path integral in
  $\mathbb{R}^3$.  Students might not be familiar with the fact that
  the vertical axis has no physical interpretation: there is no
  meaningful sense in which the spatial component of the Schwarzschild
  metric is a submanifold of a higher-dimensional Euclidean space}
\label{flamm}
\end{figure}

\begin{figure}[p] % Created by "closest_approach.R"
\centering
\includegraphics[width=\linewidth]{closest_approach_nonselfintersecting.pdf}
\caption{Light inextensible strings under tension close to a black
  hole.  Here we see non-self intersecting strings arranged by
  increasing distance of closest approach to the event horizon,
  occurring tangentially at~$\phi=0$}
\label{closest_approach_non_self_intersecting}
\end{figure}

\begin{figure}[p] % created by "closest_approach3.R"
\centering
\includegraphics[width=\linewidth]{closest_approach_selfintersecting_onegreen.pdf}

\caption{Light inextensible strings under tension close to a black
  hole.  Here, self-intersecting strings are shown for $\phi\geqslant
  0$ for clarity, but a single string is shown in green for
  $-3.69\leqslant\phi\leqslant 3.69$, making the self-intersection
  evident (all strings in the diagram, whether blue or green,
  self-intersect at $\phi=\pm\pi$, even if the point of intersection
  is far to the right).  Closest approach to the event horizon occurs
  at $\phi=0$ (3 o'clock) and $r/2M\in\left(1,1.0761\right)$; all
  strings are symmetrical about $\phi=0$}
\label{closest_approach_self_intersecting}
\end{figure}

\begin{figure}[p] % Created by "light_closest_approach.R"
\centering
\includegraphics[width=\linewidth]{light_closest_approach.pdf}
\caption{Null geodesics in the Schwarzschild geometry, tangential
  at~$\phi=0$.  Note the differences between these curves and the taut
  strings shown elsewhere: unlike taut strings, null geodesices may
  cross the event horizon inwards, and are never tangential to the
  event horizon}
\label{light_closest_approach}
\end{figure}

\begin{figure}[p]
\centering
\includegraphics[width=\linewidth]{angle_at_r_equals_2.pdf} % created by "angle_at_r_equals_2.R"
\caption{Light inextensible strings under tension close to a black
  hole.  Here strings passing through $(4M,0)$ at different angles are
  shown}
\label{strings_r_equals_2}
\end{figure}

\begin{figure}[p]
\centering
\includegraphics[width=\linewidth]{light_start_at_r_equals_2.pdf} % created by "angle_start_at_r_equals_2.R"
\caption{Null geodesics passing through $(4M,0)$ at different angles,
  shown for $0\leqslant\phi\leqslant 2\pi$}
\label{light_r_equals_2}
\end{figure}

\end{document}

