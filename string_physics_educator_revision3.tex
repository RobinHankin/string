% Authors may submit in single column format.
% Once your paper got accepted, the publisher will reformat it according to TPE house style (two column).
%\documentclass[wsdraft]{ws-tpe}

\documentclass{ws-tpe}
\usepackage{multicol}
\usepackage{units}
\usepackage{xcolor}
\usepackage[superscript]{cite}
\begin{document}


\markboth{Robin K. S. Hankin}{Light inextensible strings (thread) in the Schwarzschild metric}

\title{Light inextensible strings (thread) under tension in the
  Schwarzschild geometry}

\author{Robin K. S. Hankin}

\address{School of Engineering, Computing, and Mathematical Sciences\\Auckland University of Technology\\55 Wellesley Street East\\
  Auckland CBD, Auckland 1010, New Zealand\\
\email{hankin.robin@gmail.com}}

\maketitle

\begin{history}
\received{Day Month Year}
\revised{Day Month Year}
\end{history}


\begin{abstract}
Light inextensible string under tension is a stalwart feature of
elementary physics.  Here I show how considering such string in the
vicinity of a black hole, with the help of computer algebra systems,
can generate insight into the Schwarzschild geometry in the context of
an undergraduate homework problem.  Light taut strings minimize their
proper length, given by integrating the spatial component of the
Schwarzschild metric along the string.  The path itself is given by
straightforward numerical solution to the Euler-Lagrange equations.
If the string is entirely outside the event horizon, its closest
approach to the singularity is tangential.  At this point the string
is {\em visibly} curved, surely a memorable and informative insight.
The geometry of the Schwarzschild metric induces some interesting
nonlocal phenomena: if the distance of closest approach is less than
about $1.0761\cdot 2M$, the string self-intersects, even though it is
everywhere under tension.  Light taut strings furnish a third
interpretation of the concept ``straight line'', the other two being
null geodesics and free-fall world lines.
\end{abstract}

\keywords{Black holes; Schwarzschild metric; light inextensible
  string; minimal-length path}

\section{Introduction}

General relativity is a topic of long-standing fascination for physics
undergraduates~\cite{christensen2012}.  Students appreciate everyday
illustrations of the concepts involved, as recent educational work on
gravitational wave visualisations attests~\cite{overduin2018}.  Here I
interpret a straight line as the path adopted by a stationary, light
inexensible string under tension passing between two points.  The
analysis presented here would be suitable for an undergraduate
homework problem~\cite{romano2019}.

Considering only stationary spacetimes, we define the proper length of
a string to be the result of integrating the spatial part of the
metric tensor along string's path between its endpoints.  A light
string's being under tension means that its path will be an extremum
of proper length: non-extremal paths between two fixed points cannot
be in stable equilibrium as the tension acts to shorten the path;
stable paths must therefore be of extremal proper length.  In flat
space, the calculus of variations proves that the path is straight in
the sense of zero curvature.  In terrestrial gravity, standard
$25\times 10^{-6}\unit{kg}\cdot\unit{m}^{-1}$ cotton sewing thread at
its breaking strain of about $40\unit{N}$ would have a sag of~${\sim}
6\times 10^{-6}\unit{m}$ over a $1\unit{m}$ horizontal scale: barely
perceptible.  How does such a quotidian observation translate into the
vicinity of a black hole?  Would the ``warping of space'' result in
{\em visibly} curved thread?

Of course, no real string is perfectly weightless or inextensible.
With tension~$T$, gravitational acceleration $g$ and (proper) mass per
(proper) unit length~$\mu$, then a string will behave as though light
on lengthscales $L\lesssim T/\left(\mu g\right)$.  Using
$c^4/\left(4GM\right)$ for the strength of gravity at the event
horizon of a black hole with mass $M$ gives $L\lesssim TM/\left(\mu
c^4\right)$.  For the string to behave as though light on lengthscales
comparable with the Schwarzschild radius would require $T\gtrsim\mu
c^2$.  Exceeding this criterion would violate the dominant energy
condition (and in this context we note that transverse waves would
propagate at speed~$\sqrt{T/\mu}\sim c$).  Nevertheless, the
idealization of a string as light and inextensible is an interesting
and informative limit: Rindler~\cite{rindler}, for example, considers
lowering a heavy object to the event horizon on the end of such a
string to illustrate energetic arguments in GR.

\section{The Schwarzschild metric}

The vacuum Einstein field equations for a static, spherically
symmetric mass are solved by the Schwarzschild
metric~\cite{schwarzschild1916}.  In units in which~$G=c=1$, this is:

\begin{equation}\label{schwarzschild}
ds^2= -dt^2\left(1-2M/r\right) +\frac{dr^2}{1-2M/r} + r^2\left(d\theta^2 + \sin^2\theta d\phi^2\right).
\end{equation}

\noindent Here I consider the spatial component~$ds^2=
dr^2/\left(1-2M/r\right) + r^2d\phi^2$, confined to equatorial
points~($\theta=\pi/2$); we understand~$r\geqslant 2M$ throughout.  A
circular string at constant~$r$ will, trivially, have proper
length~$2\pi r$.  Considering first the radial component we may
calculate the proper length $\ell\left(r_1,r_2\right)$ of a string
stretching from~$r=r_1\geqslant 2M$ to~$r_2\geqslant r_1$ at constant
$\theta,\phi$:


% \begin{equation}\label{radial_string_length_properunits}
%   \int_{r=r_1}^{r_2}\frac{dr}{\sqrt{1-1/r}}=
%   \left.
%   \sqrt{r(r-1)} +\frac{1}{2}\log\left(
%   \frac{\sqrt{r}+\sqrt{r-1}}{\sqrt{r}-\sqrt{r-1}}\right)
%   \right|_{r_1}^{r_2}
%   \end{equation}

\begin{equation}\label{radial_string_length}
  \int_{r=r_1}^{r_2}\frac{dr}{\sqrt{1-2M/r}}=
  \left.
  \sqrt{r(r-2M)} +M\log\left(
  \frac{\sqrt{r}+\sqrt{r-2M}}{\sqrt{r}-\sqrt{r-2M}}\right)
  \right|_{r_1}^{r_2}.
  \end{equation}

Observe that integration may be performed with a lower limit
of~$r_1=2M$ with no difficulty, even though the radial component of
the Schwarzschild metric approaches infinity there.  Elementary
methods show that the proper length $\ell$ of radial strings has some
non-Euclidean features.  Firstly, consider the proper length from
$r=2M$ to $r=2M(1+w)$ where $w$ is small, $0<w\ll 1$.  Then

%  \begin{equation}\label{ell_properunits}
%    \ell=\int_{r=1}^{r=1+w}ds
%    =2\sqrt{w} + \frac{1}{3}w^{3/2} + O(w^{5/2})
%  \end{equation}

 \begin{equation}\label{ell}
   \ell =
   \int_{r=2M}^{r=2M(1+w)}ds
   =2M\left(2\sqrt{w} + \frac{1}{3}w^{3/2} + O(w^{5/2})\right).
 \end{equation}

(compare the flat space string length of $2Mw$).  This would indicate
 that, close to the Schwarzschild radius $r=2M$, more string is
 required to connect two points with slightly differing radial
 coordinates than would be expected classically.  Taking a solar-mass
 black hole and $w=10^{-3}$ as an example, the proper length computed
 in flat space would be $10^{-2}\cdot 2M\simeq 3\,\mathrm{m}$; compare
 $\simeq 187\,\mathrm{m}$ from equation~\ref{ell}.  Passing to the
 case $w\gg 1$ we have an asymptotic expansion that begins

% \begin{equation}\label{asymptotic_ell_properunits}
%   \ell = w +  \frac{1}{2}\log w + \frac{1}{2} + \log(2)  +  \frac{1}{8}w^{-1} + O(w^{-2}),
% \end{equation}
% 
\begin{equation}\label{asymptotic_ell}
  \ell = 2M\left(w +  \frac{1}{2}\log w + \frac{1}{2} + \log(2)  +  \frac{1}{8}w^{-1} + O(w^{-2})\right).
\end{equation}


\noindent The discrepancy between the flat-space result of $2Mw$ and
equation~\ref{asymptotic_ell} suggests that, even far from the event
horizon, radial strings are arbitrarily longer than flat space would
imply: a nice illustration of the failure of Euclidean geometry.
Again taking a solar mass black hole but with $w=2\times 10^7$
(corresponding to the orbital radius of Mercury) the difference
between the classical result and equation~\ref{asymptotic_ell} is
about $28\,\mathrm{km}$ of extra string.  However, the prediction is
unlikely to be the basis of a fourth test of general relativity---the
others being Shapiro time delay, bending of starlight, and the
perihelion advance of Mercury~\cite{einstein1916}---as the Sun is not
inside its own Schwarzschild radius.

\section{Nonradial string}
Integrating along a path between two
points~$p_1=\left(r_1,\phi_1\right)$ and~$p_2=\left(r_2,\phi_2\right)$
gives the {\em proper} length of the path, which is the length of an
inextensible string joining~$p_1$ and~$p_2$.  A taut but light string
from~$p_1$ to~$p_2$ will adopt an extremal-length path.  Such paths
may be found by the calculus of variations; parametrizing a curve in
terms of~$r=r\left(\phi\right)$ gives us a path length of

% \begin{equation}
% %  \int_{p_1}^{p_2}\sqrt{\frac{\left(r'\right)^2}{1-2M/r} + r^2}d\phi=
%   \int_{p_1}^{p_2}\left(\frac{\left(r'\right)^2}{1-1/r} + r^2\right)^\frac{1}{2}d\phi=
%   \int_{p_1}^{p_2}F\left(r,r'\right)d\phi
% \end{equation}

\begin{equation}
%  \int_{p_1}^{p_2}\sqrt{\frac{\left(r'\right)^2}{1-2M/r} + r^2}d\phi=
  \int_{p_1}^{p_2}\left(\frac{\left(r'\right)^2}{1-2M/r} + r^2\right)^\frac{1}{2}d\phi=
  \int_{p_1}^{p_2}F\left(r,r'\right)d\phi
\end{equation}

\noindent where dashes denote differentiation with respect to~$\phi$.
The Euler-Lagrange equations for this system,
$\frac{d}{d\phi}\frac{\partial F}{\partial r'}-\frac{\partial
  F}{\partial r}=0$ give an expression for the second derivative
of~$r$ (Mathematica idiom would be {\tt EulerEquations[Sqrt[y[x]\^{}2
      + y'[x]\^{}2/(1-1/y[x])],y[x],x]}):

\begin{equation}\label{rdashdash}
  r''\left(\phi\right) =
  (r-2M) + \frac{(2r-3M)\left(r'\right)^2}{r\left(r-2M\right)},\qquad r>2M.
\end{equation}

\noindent Solutions to equation~\ref{rdashdash} correspond to a taut
string confined to Flamm's paraboloid~\cite{flamm1916}; see
Figure~\ref{flamm}.  It is interesting to compare
equation~\ref{rdashdash} with the corresponding equation for
equatorial null geodesics, which is usually given~\cite{wald} in terms
of $u=2M/r$ as $u''=3u^2/2-u$, but is equivalent
to~$r''\left(\phi\right)=r-3M+2\left(r'\right)^2/r$.  This shows that
light, being sensitive to the time component of the metric, behaves
differently from string, which is not (at this point it might be worth
``reminding'' students that a straight line in flat space,
$r=r_0\sec\left(\phi-\phi_0\right)$, obeys
$r''=r+2\left(r'\right)^2/r$).

\begin{figurehere} % Created by "flamm.R"
\centering
\includegraphics[width=\linewidth]{flamm_string.pdf}
\caption{Perspective view of Flamm's paraboloid with superimposed minimal-length
  path corresponding to a taut, light string}
\label{flamm}
\end{figurehere}

There does not appear to be a simple analytical solution to
equation~\ref{rdashdash}; but it is a straightforward second order
ODE, amenable to numerical solution and no harder than the equivalent
for null geodesics, a standard homework problem.  Here we consider
only strings exterior to the event horizon, $r>2M$, which have a
well-defined point of closest approach to the singularity.  Because
the system is invariant under~$\phi\longrightarrow-\phi$, the string
must be symmetrical about this point.  If $r'$ is small, then a string
follows $r''=r-2M$, while tangential null geodesics follow $r''=r-3M$.
Working to first order, and writing~$d$ for the difference in
Schwarzschild~$r$ between string and null geodesic, we have $d''=M$.
Transforming coordinates to $(x,y) = (r\phi,r-r_0)$ and with boundary
conditions corresponding to both string and null geodesic being
tangential at the $(x,y)$ origin, we see that
$d=\frac{M}{2}(x/r_0)^2$.  However, a static local observer tied to
the string would measure
$d_\mathrm{local}=d/\sqrt{1-2M/r_0}$---applying
equation~\ref{schwarzschild}---and obtain
$d_\mathrm{local}=\frac{M}{2}(x/r_0)^2\left(1-2M/r_0\right)^{-1/2}$
which would correspond to a radius of curvature
of~$r_0\cdot\frac{r_0}{M}\cdot\left(1-2M/r_0\right)^{-1/2}$.  Thus a
taut string would {\em look} curved to a static observer in
Schwarzschild coordinates, in the sense that a tangential light beam
and taut string would diverge quadratically.  At the photon sphere
$r_0=3M$, such a string would appear to have a radius of~$\sqrt{27}M$.
For a solar mass black hole this would be about $15\unit{km}$, a
readily understandable result.


\section{Numerical results}

Here I solve equation~\ref{rdashdash} numerically, using the R
programming language~\cite{rcore2019} and the \verb+deSolve+ package
for ordinary differential equations~\cite{soetart2010}.  First we
consider strings for which $r>2M$ everywhere (that is, completely
exterior strings).
Figure~\ref{closest_approach_non_self_intersecting} shows a sequence
of non-self-intersecting numerical solutions to
equation~\ref{rdashdash}, rotated so that the tangential point of
closest approach occurs at~$\phi=0$.
Figure~\ref{closest_approach_self_intersecting} shows the
corresponding diagram for self-interesecting strings
(Figure~\ref{light_closest_approach} shows equivalent null geodesics
for comparison).  A bisection technique reveals that if the closest
approach is greater than a critical value of about~$\simeq 1.0761\cdot
2M$, the string does not self-intersect; if the closest approach is
less than this value, then the string crosses itself at~$\phi=\pi/2$.

\begin{figurehere} % Created by "closest_approach.R"
  \centerline{
    \includegraphics[width=\linewidth]{closest_approach_nonselfintersecting.pdf}
  }
  \caption{Light inextensible strings under tension close to a black
    hole.  Here we see non-self intersecting strings arranged by
    increasing distance of closest approach to the event horizon,
    occurring tangentially at~$\phi=0$}
  \label{closest_approach_non_self_intersecting}
\end{figurehere}

\begin{figurehere} % created by "closest_approach2.R"
  \centerline{
    \includegraphics[width=\linewidth]{closest_approach_selfintersecting.pdf}
  }
  \caption{Light inextensible strings under tension close to a black
    hole.  Here self-intersecting strings are shown for $\phi\geqslant
    0$.  Closest approach to the event horizon occurs at $\phi=0$ (3
    o'clock) and $r/2M\in\left(1,1.0761\right)$; strings are symmetrical
    about $\phi=0$ but for clarity only $\phi\geqslant 0$ is shown}
  \label{closest_approach_self_intersecting}
\end{figurehere}

\begin{figurehere} % Created by "light_closest_approach.R"
  \centerline{
    \includegraphics[width=\linewidth]{light_closest_approach.pdf}
  }
  \caption{Null geodesics in the Schwarzschild geometry, tangential
    at~$\phi=0$.  Note the differences between these curves and the taut
    strings shown elsewhere: unlike taut strings, null geodesices may
    cross the event horizon inwards, and are never tangential to the
    event horizon}
  \label{light_closest_approach}
\end{figurehere}

There are other ways to parametrize completely exterior strings;
Figure \ref{strings_r_equals_2}, for example, shows strings passing
through~$(4M,0)$.  Figure~\ref{light_r_equals_2} shows light paths for
comparison.

\begin{figurehere}
  \centerline{
    \includegraphics[width=\linewidth]{angle_at_r_equals_2.pdf} % created by "angle_at_r_equals_2.R"
  }
  \caption{Light inextensible strings under tension close to a black
    hole.  Here strings passing through $(4M,0)$ at different angles are
    shown}
  \label{strings_r_equals_2}
\end{figurehere}

\begin{figurehere}
  \centerline{
    \includegraphics[width=\linewidth]{light_start_at_r_equals_2.pdf} % created by "angle_start_at_r_equals_2.R"
  }
  \caption{Null geodesics passing through $(4M,0)$ at different angles,
    shown for $0\leqslant\phi\leqslant 2\pi$}
  \label{light_r_equals_2}
\end{figurehere}

\section{Discussion}

The concept of ``straight line'' is problematic in general relativity.
Although photon paths provide a definition of straightness, their
world lines are sensitive to the time component of the metric.  Light
taut strings respond only to the spatial components of the metric, and
their configuration is arguably a more intuitive and helpful way to
think about straightness.  Using numerical methods to compute string
configuration is suitable for an undergraduate homework problem.

\bibliographystyle{unsrt}
\bibliography{stringrefs.bib}
\end{document} 
